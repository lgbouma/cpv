% upon AAS submission
%\documentclass[12pt,twocolumn,tighten,linenumbers]{aastex63}
%\documentclass[12pt,twocolumn,tighten,linenumbers,trackchanges]{aastex63}
% drafting / arxiv
\documentclass[11pt,twocolumn,tighten]{aastex63}
\turnoffedit

\usepackage{apjfonts}
\usepackage{url}
\usepackage{hyperref}
\usepackage{natbib}
\usepackage{amsmath,amstext,amssymb}
\usepackage[caption=false]{subfig} % for subfloat
\usepackage{xcolor, fontawesome}
\usepackage{color}

\newcommand{\rprs}{{$R_p/R_{\star}$}}
\newcommand{\vsini}{{$V \sin i$}}
\newcommand{\teff}{T$_{\textrm{eff}}$}
\newcommand{\kms}{{km\,s$^{-1}$}}
\newcommand{\gcc}{{g\,cm$^{-3}$}}
\newcommand{\rstar}{{$R_\star$}}
\newcommand{\rhostar}{{$\rho_\star$}}
\newcommand{\mearth}{{M$_\oplus$}}
\newcommand{\rearth}{{R$_\oplus$}}
\newcommand{\rsun}{{R$_\odot$}}
\newcommand{\msun}{{M$_\odot$}}

\newcommand{\minus}{\scalebox{0.5}[1.0]{$-$}}

\newcommand{\eg}{e.g.,} 

\newcommand{\mstype}{letter}

%%%%%%%%%%%%%%%%
% INSTITUTIONS %
%%%%%%%%%%%%%%%%
\newcommand{\caltech}{Department of Astronomy, MC 249-17, California Institute of Technology, Pasadena, CA 91125, USA}
\newcommand{\mitkavli}{MIT Kavli Institute and Department of Physics, 77 Massachusetts Avenue, Cambridge, MA 02139}
\newcommand{\berkeley}{Astronomy Department, University of California, Berkeley, CA 94720, USA}

%
% ms specific numbers
%

%%%%%%%%%
% STARS %
%%%%%%%%%
\newcommand{\nstarssearched}{{XXYYY}}

%%%%%%%%
% CPVS %
%%%%%%%%
\newcommand{\ncpvsfound}{{48}}
\newcommand{\nmaybes}{{19}}


\begin{document}

\title{Bright and Nearby Complex Periodic Variables from Four Years of TESS}

\correspondingauthor{TBD}
\email{TBD}

\received{---}
\revised{---}
\accepted{---}
\shorttitle{Cycle 1-4 CPVs} 

\shortauthors{TBD}

%%%%%%%%%%%%%%%%%%%%%%%%%%%%%%%%%%%%%%%%%%
%%%%%%%%%%%%%%%%%%%%%%%%%%%%%%%%%%%%%%%%%%
%%%%%%%%%%%%%%%%%%%%%%%%%%%%%%%%%%%%%%%%%%

\author{Authors to include (ORDER TBD)}

% primary authors
\author[0000-0002-0514-5538]{Luke~G.~Bouma}
\altaffiliation{51 Pegasi b Fellow}
\affiliation{\caltech}

\author[0000-0002-7778-3117]{Rahul~Jayararaman}
\affiliation{\mitkavli}

% [to be confirmed after these]
\author{Saul~Rappaport}
\affiliation{\mitkavli}

\author{Lynne~A.~Hillenbrand}
\affiliation{\caltech}

\author[0000-0003-2058-6662]{George~R.~Ricker}
\affiliation{\mitkavli}

%%%%%%%%%%%%%%%%%%%%%%%%%%%%%%%%%%%%%%%%%%
%%%%%%%%%%%%%%%%%%%%%%%%%%%%%%%%%%%%%%%%%%
%%%%%%%%%%%%%%%%%%%%%%%%%%%%%%%%%%%%%%%%%%

\begin{abstract}
  Complex periodic variables (CPVs) are pre-main-sequence M-dwarfs
  with periodic optical modulation that includes both sharp and broad
  troughs.
  The troughs are usually superposed on smooth, spot-like modulation.
  Based on a lack of observed infrared excesses, these objects do not
  have primordial gas-rich disks, or even particularly massive debris disks.
  Nonetheless, the most plausible explanations for their optical
  variability invoke long-lived material that orbits at the corotation
  radii, for instance transiting dust clumps or stellar prominences.
  Here, we report new CPVs discovered in the 120-second cadence TESS data
  collected between July~2018 and Sep~2022 (Sectors 1-55).
  Our search used two methods, the first based on counting dips in
  phase-folded light curves, and the second based on counting peaks in
  the Fourier domain.
  From our analysis of \nstarssearched\ M-dwarfs with $T$$<$16 and
  $d$$<$150\,pc, we find \ncpvsfound\ gold-standard CPVs.
  Most of these discoveries are new, and they include the brightest
  ($T$$\approx$10) and closest ($d$$\approx$20\,pc) examples of this
  class of variable star known.
  A few objects are ``outliers among outliers''; for
  instance LP 12-502 shows up to eight local minima per cycle, exhibited
  sudden shape ``switches'' over less than one cycle, yet maintained
  a near-constant cycle duration over more than one thousand cycles.
  The general cause of the optical variability for these objects remains
  enigmatic.  However our sample is a step toward the spectroscopic and
  demographic studies that may ultimately resolve the mystery of what
  physically explains these objects.
  Connections to the broader context of star, disk, and
  exoplanet evolution are emphasized.
\end{abstract}

\keywords{Weak-line T Tauri stars (1795), Periodic variable stars
(1213), Circumstellar matter (241), Star clusters (1567), Stellar
magnetic fields (1610), Stellar rotation (1629)}

\section{Introduction}
\label{sec:intro}

All pre-main-sequence stars vary in optical brightness, and the origin
of such variability is, in most cases, understood.  Well-explored
sources of optical variability include inhomogeneities on stellar
surfaces such as starspots and faculae
\citep[e.g.][]{2021isma.book.....B}, occultations by gas-rich
circumstellar disks \citep[e.g.][]{2017MNRAS.470..202B}, and, in
geometrically favorable circumstances, eclipses by stars and planets.

Data from K2 and TESS have yielded a new class of variable star whose
root cause is not yet qualitatively understood: complex periodic
variables (CPVs).  These objects are identified phenomenologically
using their optical light curves, which show periodic sharp and
sometimes broad troughs, often superposed on smooth spot-like
modulation.  Some of the objects show up to eight local minima (dips)
per cycle; sharp peaks are also common.  Most of these variable stars
turn out to be pre-main-sequence M-dwarfs without near-infrared
excesses, with ages of $\approx$5-150 million years (Myr), and
rotation periods of at most two days; they are observed to be
$\approx$1\% of M-dwarfs younger than 100 Myr
\citep{2017AJ....153..152S,2018AJ....155...63S,2019ApJ...876..127Z,2022AJ....163..144G}.
The dips can be chromatic, with a reddening law plausibly consistent
with dust
\citep{2020AJ....160...86B,2022AJ....163..144G,2023MNRAS.518.2921K}.
And finally, while the dip shapes can ``jump'' between different
depths and durations over less than one cycle, they more often evolve
gradually, over tens to hundreds of cycles
\citep[e.g.][]{2017AJ....153..152S,2020AJ....160...86B,2023ApJ...945..114P}.

The issue of what exactly causes the variability that defines the CPVs
has yet to be resolved.  There are a few competing explanations.  All
young M-dwarfs are spotted, which produces flux variations over
characteristic timescales of the rotation period, $P_{\rm rot}$, and
its half-harmonic, $0.5\,P_{\rm rot}$.  The observed dips occur over
durations as short as $0.05\,P_{\rm rot}$; a ``starspot-only''
scenario is therefore highly unlikely for any of the objects with
sufficiently sharp dips
\citep{2017AJ....153..152S,2021MNRAS.500.1366K}.   The more likely
scenarios instead invoke sharp geometries with material extrinsic to
the stellar surface \citep[see
Refs.~][]{2017AJ....153..152S,2022AJ....163..144G}.  In the ``clump''
scenario, opaque dust clumps orbiting near the Keplerian corotation
radius eclipse the star
\citep{2017AJ....153..152S,2023MNRAS.518.4734S}.  In the
``prominence'' scenario, long-lived condensations of plasma are
trapped along the star’s magnetic field lines, also near corotation
\citep{2022MNRAS.514.5465W}.  And in a ``screen'' scenario, the inner
wall of a quiescent circumstellar disk blocks a portion of the stellar
surface to produce sudden dips whenever spots come into view
\citep{2019ApJ...876..127Z}.  In our view, the arguments seem best
established for the dust clump and prominence hypotheses
\citep{2023MNRAS.518.4734S,2022MNRAS.514.5465W}.  However, unambiguous
evidence has yet to be presented to demonstrate which, if any, of
these scenarios is correct.

It has not yet been possible to rule between the candidate mechanisms
CPVs have historically been both hard to discover and hard
to characterize.   Discoverability has been linked to rarity: 
CPVs comprise about 1\% of the youngest 1\% of M-dwarfs
\citep{2018AJ....155..196R}.  Out of the millions of stars monitored
by K2 and TESS, about 50 CPVs have been reported to date
\citep{2017AJ....153..152S,2018AJ....155...63S,2019ApJ...876..127Z,2022AJ....163..144G,2023ApJ...945..114P}.
The known CPVs are correspondingly faint; the initial K2 discoveries
\citep{2017AJ....153..152S} were M2-M6 dwarfs at distances
$\gtrsim$100\,pc, yielding optical brightnesses of $V$$\approx$15.5 to
$V$$>$20.  This renders time-series optical spectroscopy at high
resolution technically impossible, despite its potential utility in
ruling between the models.

One way to help rule between the mechanisms is to therefore find
bright and nearby CPVs, since these objects will be the most amenable
to detailed photometric and spectroscopic analyses.  To do this, in
this work, we use 120-second cadence data acquired by the Transiting
Exoplanet Survey Satellite between July~2018 and Sep~2022 (Sectors
1-55; Cycles 1-4).  We present our search methods in
Section~\ref{sec:methods}; and the properties of the resulting CPV
catalog in Section~\ref{sec:results}.  Open questions are discussed in
Section~\ref{sec:discussion}, and we conclude in
Section~\ref{sec:conclusion}.




\section{Methods}
\label{sec:methods}

%We will first describe our stars of interest
%(Section~\ref{subsec:selectionfn}), and then describe our methods for
%identifying which of those stars appear as CPVs
%(Section~\ref{subsec:discoverymethods}).

\subsection{Stellar selection function}
\label{subsec:selectionfn}

We chose to analyze only the ``short'' 120-second cadence data
acquired by TESS between July~2018 and Sep~2022 (Sectors 1-55).
Specifically, we used the 120-second cadence light curve products
produced by the mission's Science Processing and Operations Center at
NASA Ames \citep{2016SPIE.9913E..3EJ}.  While the TESS data products
also include full frame images with cadences of 200, 600, and 1800
seconds, we limited our scope in this work for the sake of simplicity
in data handling.  In exchange, we lose out in both completeness and
homogeneity of the selection function.  While TESS cumulatively
observed $\approx$90\% of the sky for at least one lunar month between
July~2018 and Sep~2022, the 120-second cadence data were acquired for
only a pre-selected set of stars over Sectors 1-26, and then a
guest-investigator driven set of stars over Sectors 27-55
\citep{2021PASP..133i5002F}.

To assess the completeness of the resulting 120-second cadence data that we
analyze, we cross-matched TIC8 \citep{2018AJ....156..102S} against the
Gaia DR2 point-source catalog \citep{2018A&A...616A...1G}.   We opted
for Gaia DR2 rather than DR3 because the base catalog for TIC8 was
Gaia DR2, which facilitated a one-to-one crossmatch using the Gaia source
identifiers.  This exercise showed that for $T$$<$16 M-dwarfs, the TESS
2-minute data are roughly 50\% complete at $\approx$50\,pc.  At
$<$20\,pc, $\gtrsim$80\% of the $T$$<$16 M-dwarfs have at least one
sector of short-cadence data; at $>$100\,pc, $\lesssim$10\% of such
M-dwarfs have at least one sector of short-cadence data.  Armed
with this understanding, we then used
our cross-match between Gaia DR2 and TIC8 to select our stars of
interest, which we defined as stars with 120-second cadence TESS light
curves that satisfied
\begin{align}
  T &< 16\\
  G_{\rm BP} - G_{\rm RP} &> 1.5 \\
  M_{\rm G} &> 4 \\
  d &< 150\,{\rm pc},
\end{align}
for $M_{\rm G} = G + 5\log(\varpi_{\rm as}) + 5$ the Gaia $G$-band
absolute magnitude, $\varpi_{\rm as}$ the parallax in units of
arcseconds, and a geometric distance $d$ defined by inverting the parallax
and ignoring any small zero-point corrections.
This selection function includes dwarf stars later than spectral types
of $\approx$K6V.  Hereafter, we will refer to this set of stars as the
target sample, and it includes \nstarssearched\ stars with (roughly)
M-dwarf spectral type, down to $T$$<$16 and out to $d$$<$150\,pc.

\begin{figure*}[!t]
	\begin{center}
		\subfloat{
			\includegraphics[width=0.8\textwidth]{lc_mosaic_fav3.pdf}
		}
		
		\subfloat{
			\includegraphics[width=0.8\textwidth]{models.pdf}
		}
		%\subfloat{
		%	\includegraphics[width=0.8\textwidth]{flux_vs_time_xmin0000_xmax0010.pdf}
		%}
	\end{center}
	\caption{
		{\bf Complex periodic variables (CPVs)}:
		{\it Top:} Phase-folded TESS light curves of three CPVs.  Each is
		stacked over one month.  Gray are raw 2-minute data; black bins to
		300 points per cycle.  Periods in hours are in the bottom right of
		each panel.  In order left-to-right, the objects are LP 12-502
		(TIC 402980664; Sector~19), TIC 94088626 (Sector 10), and TIC
		425933644 (Sector~28).
		{\it Bottom:} Plausible cartoon models for the phenomenon.
		%{\it Bottom:} Flux {\it vs.}~time showing 10 cycles for each
		%object.  Raw 2-minute data are the gray background; the black line
		%is drawn between points binned to 1/100$^{\rm th}$ of a cycle.
	}
	\label{fig:f1}
%	\vspace{-0.5cm}
\end{figure*}


\subsection{CPV discovery}
\label{subsec:discoverymethods}

Previous methods that have yielded CPV discoveries include visually
examining stars known to be in young clusters
\citep{2017AJ....153..152S}, and automatically flagging rapid rotators
with a large number of strong Fourier harmonics
\citep{2019ApJ...876..127Z}.  The latter approach is followed by
visual vetting, since most rapid rotators with many Fourier harmonics
are false positives such as eclipsing binaries or multiple stars
blended into a single photometric aperture.  In this work, we adopted
the Fourier approach,
and also implemented a new search approach based on counting the
number of sharp local minima in phase-folded light curves.  We then
applied these two search techniques independently.   


\subsubsection{Fourier analysis}
\label{subsec:fourier}
For the Fourier analysis, we followed
\citet{2019ApJ...876..127Z}.
{\bf todo for Rahul or Saul: explain the approach, in a few
paragraphs.  Was the SAP\_FLUX or PDCSAP used? etc. }


\subsubsection{Counting dips}
\label{subsec:counting}

The dip counting technique aims to count sharp local minima in
phase-folded light curves.  CPVs will preferably have at least three
such minima in order to be distinct from false positives such as
synchronized and spotted binaries (``RS CVn'' stars). 

For our dip-counting pipeline, we began with the PDC\_SAP flux for
each sector, removed non-zero quality flags, and normalized the light
curve to one by dividing out its median value.  We then flattened the
light curve using a 5-day sliding median filter, as implemented in
\texttt{wotan} \citep{2019AJ....158..143H}.  On the resulting
cleaned and flattened light curve, we ran a periodogram search, opting
for the \citet{1978ApJ...224..953S} phase dispersion minimization
(PDM)
algorithm implemented in \texttt{astrobase}
\citep{2021zndo...1011188B} due to its shape agnosticism.  If a period
below 2 days was identified, we reran the periodogram at a
finer grid to improve the accuracy of the period determination.

Once a star's period $P$ was identified, we binned the phased light curve
to 100 points per cycle.  To separate ``sharp'' local minima from
smooth spot-induced variability, we then iteratively fit penalized
splines to the wrapped phase-folded light curve, excluding points more
than two standard deviations away from the local continuum
\citep{2019AJ....158..143H}.  The maximum number of equidistant spline
knots per cycle is the parameter in this framework that controlled the
meaning of ``sharp'' --- we allowed at most 10 such knots per cycle,
though for most stars fewer knots were preferred based on an
$\ell^2$-norm penalty. 

We then identified local minima in the resulting residual light curve
using the SciPy \texttt{find\_peaks} utility (CITE), which is based on
comparing adjacent
values in an array.  For a peak to be flagged as significant, we
required it to have a width of at least $0.02\,P$, and a height of at
least twice the point-to-point RMS.  This latter quantity is defined as
the 68$^{\rm th}$ percentile of the distribution of the residuals from
the median value of $\delta f_i \equiv f_i - f_{i+1}$, where $f$ is
the flux and $i$ is an index over time.

To correctly identify local minima near the edges of the phased light
curve, which usually would cover phases $\phi \in [ 0,1 ]$, we in fact
performed the entire procedure over a phase-folded light curve
spanning $\phi \in [-1,2 ]$, by duplicating and concatenating
the ordinary phase-folded light curve.
The free parameters we adopted throughout this analysis procedure, for
instance the maximum number of spline knots per cycle, and how large
and wide of a local minimum to consider a ``true dip'', were chosen
during a testing period based on their ability to correctly re-identify 
a large fraction ($>$90\%) of known CPVs, while also being able to
consistently reject common false positives such as rapidly rotating
spot-induced variability and typical eclipsing binaries.

Overall, for a star to clear this process and to proceed to manual
examination, we required that it have a peak PDM period below two
days, and that it exhibited at least three sharp local minima (as
algorithmically reported) in at least one observed TESS sector.


\subsection{Manual vetting}

We visually assessed whether the objects found using the Fourier
(Section~\ref{subsec:fourier}) and dip-counting
(Section~\ref{subsec:counting}) techniques were consistent with 
expectations for CPVs by assembling 
the data shown in Appendix~\ref{app:vetting}.
The false-positive rates from our dip-counting approach, as-implemented,
%TODO: left off here


While
the highest S/N cases were generally found by both the Fourier and dip-counting
techniques, we
generally found that 




\section{Results}
\label{sec:results}



\begin{figure*}[!tp]
	\begin{center}
    \centering
    \includegraphics[width=\textwidth]{lc_mosaic_dlt150_good_all.pdf}
		\caption{
			{\bf CPVs from a search of the TESS 2-minute data at
      $d$$<$150\,pc, acquired between July~2018 and Sep~2022.}
      Phased TESS light curves over 1 month are shown for 40 CPVs;
      they include the brightest and closest examples of CPVs known
      ($V$=14; $J$=9.5; $d$=25\,pc).  Gray are raw 2-minute data;
      black bins to 300 points per cycle.  Periods in hours are listed
      in the lower right corners of each panel.
      The TIC identification numbers and sectors are listed in
      Table~\ref{tab:thetable}.
		}
		\label{fig:revolution}
%		\vspace{-0.8cm}
	\end{center}
\end{figure*}

\begin{figure*}[!t]
	\begin{center}
			\centering
			\includegraphics[width=0.9\textwidth]{lc_mosaic_dlt150_good_changers.pdf}
		\vspace{-0.2cm}
		\caption{
		{\bf Evolution of CPVs}: ``Before and
      after'' for six CPVs from Figure~\ref{fig:revolution}.  Panels
      in the top two rows are separated by two years
      ($\approx$$10^3$ cycles); each panel shows one month.
      Periods are listed in hours. 
      {\bf todo: make these for everything.}
%			\vspace{-0.8cm}
		}
		\label{fig:evoln}
	\end{center}
\end{figure*}

\begin{figure*}[!t]
	\begin{center}
    \centering
    \includegraphics[width=0.9\textwidth]{TIC_402980664_P18.5604_2min_phase_timegroups_mosaic.pdf}
		\vspace{-0.45cm}
		\caption{
			{\it Bottom:}
      Evolution of LP 12-502 ($P$=18.6\,h) at fixed period over three
      years; small text denotes cycle number.  The TESS pointing law
      dictates time gaps; larger gaps tend to yield larger shape
      changes.
      The dips usually evolve over tens to hundreds of cycles.
      However cycles 1233-1264 show a dip that ``switched'' from a depth
      and duration of 3\% and 3\,hr to 0.3\% and 1\,hr over less than
      one cycle.
			\vspace{-0.8cm}
		}
		\label{fig:lp}
	\end{center}
\end{figure*}






\section{Discussion}
\label{sec:discussion}

%A number of auxiliary
%questions follow.  After correcting for line-of-sight inclination, are
%most young M-dwarfs CPVs?  What observational signatures distinguish
%the proposed models (Figure~\ref{fig:f1}, bottom row)?  What
%organizational regularities characterize the CPV as a class of
%variable star?  What sets the extremes of the CPV population, such as
%the longest rotation periods, hottest stellar temperatures, and oldest
%stellar ages?  And finally, what connections, if any, do CPVs have to
%topics such as stellar evolution, M-dwarf magnetic fields, debris
%disks, and close-in exoplanets?


{\it Why would the prominence scenario tend toward corotation?}
This is analogous to quiescent
prominences on the Sun \citep{1967SoPh....2...39K}.

\section{Conclusions}
\label{sec:conclusion}

\acknowledgments
This work was supported by the 
Heising-Simons 51~Pegasi~b Fellowship (LGB)

{\bf author contribution statement goes here}
L.G.B.~and R.~J. conceived the project and executed the
dip-based and Fourier-based searches, respectively.
They both drafted the initial manuscript.
S.~R. vetted the results from the Fourier search.
rotation- and lithium-based age analyses, and drafted the manuscript.
L.A.H.~contributed to project execution.
All authors assisted in manuscript revision.

\clearpage
\bibliographystyle{yahapj}                            
\bibliography{bibliography} 

\appendix
\section{Validation plots}
\label{app:vetting}

Figure~\ref{fig:vet} shows the type of plot used to visually assess
whether a source was likely to be a CPV, eclipsing binary, or simply a
rapidly rotating star.

\begin{figure*}[!t]
	\begin{center}
    \centering
    \includegraphics[width=\textwidth]{402980664_S0019_120sec_cpvvetter.pdf}
		\vspace{-0.45cm}
		\caption{
      Validation plots used to label CPVs (the complete figure
      set of \ncpvsfound\ images is available online).
      Panels are as follows.
      {\it a)}: Phase-folded light curve; gray points are raw 2-minute
      data and black points are binned to 200 points per cycle.
      {\it b)}: Phase-dispersion minimization (PDM) periodogram.
      {\it c)}: DSS finder chart, with 1- and 2-TESS pixel radius
      circles displayed for scale.
      {\it d)}: Cleaned light curve, binned to 20-minute cadence.
      {\it e)}: Phase-folded light curve, binned to 100 points per
      cycle.  The gray line denotes the automated spline-fit to the
      wrapped phase-folded light curve, and small gray triangles
      denote automatically identified local minima; 
      one of the four sharp dips for this star was not identified!
      {\it f)}: Phase-folded light curve at twice the peak period.
      {\it g)}: Phase-folded light curve at half the peak period.
      {\it h)}: Phase-folded time-series within the ``background''
      aperture defined in the SPOC light curves.
      {\it i)}: Phase-folded flux-weighted centroid in the column
      direction.
      {\it j)}: Phase-folded flux-weighted centroid in the row
      direction.
      {\it k)}: Gaia DR2 color--absolute magnitude diagram.     
      {\it l)}: Information from Gaia DR2, TIC8, and the automated
      dip-counting search pipeline.  ``Neighbors'', abbreviated
      ``nbhr'', are listed within apparent distances of 2 TESS pixels
      if $\Delta T$$<$2.5.
      }
		\label{fig:vet}
	\end{center}
\end{figure*}





\clearpage
\listofchanges


\end{document}
