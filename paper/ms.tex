% upon AAS submission
%\documentclass[12pt,twocolumn,tighten,linenumbers]{aastex63}
%\documentclass[12pt,twocolumn,tighten,linenumbers,trackchanges]{aastex63}
% drafting / arxiv
\documentclass[11pt,twocolumn,tighten]{aastex63}
\turnoffedit

\usepackage{apjfonts}
\usepackage{url}
\usepackage{hyperref}
\usepackage{natbib}
\usepackage{amsmath,amstext,amssymb}
\usepackage[caption=false]{subfig} % for subfloat
\usepackage{xcolor, fontawesome}
\usepackage{color}
\usepackage{enumitem}

\newcommand{\red}{\color{red}}

\newcommand{\rprs}{{$R_p/R_{\star}$}}
\newcommand{\vsini}{{$V \sin i$}}
\newcommand{\kms}{{km\,s$^{-1}$}}
\newcommand{\gcc}{{g\,cm$^{-3}$}}
\newcommand{\rstar}{{$R_\star$}}
\newcommand{\rhostar}{{$\rho_\star$}}
\newcommand{\mearth}{{M$_\oplus$}}
\newcommand{\rearth}{{R$_\oplus$}}
\newcommand{\rsun}{{$R_\odot$}}
\newcommand{\msun}{{$M_\odot$}}
\newcommand{\bprp}{G_{\rm BP} - G_{\rm RP}}

\newcommand{\minus}{\scalebox{0.5}[1.0]{$-$}}

\newcommand{\mstype}{letter}

%%%%%%%%%%%%%%%%
% INSTITUTIONS %
%%%%%%%%%%%%%%%%
\newcommand{\caltech}{Department of Astronomy, MC 249-17, California Institute of Technology, Pasadena, CA 91125, USA}
\newcommand{\mitkavli}{MIT Kavli Institute and Department of Physics, 77 Massachusetts Avenue, Cambridge, MA 02139}
\newcommand{\berkeley}{Astronomy Department, University of California, Berkeley, CA 94720, USA}
\newcommand{\princeton}{Department of Astrophysical Sciences, Princeton University, 4 Ivy Lane, Princeton, NJ 08540, USA}
\newcommand{\esa}{European Space Agency (ESA), European Space Research and Technology Centre (ESTEC), Keplerlaan 1, 2201 AZ Noordwijk, Netherlands}
\newcommand{\howard}{Department of Physics and Astronomy, Howard University, Washington DC, 20059}
\newcommand{\goddard}{Center for Research and Exploration in Space Science and Technology, and X-ray Astrophysics Laboratory, NASA/GSFC, Greenbelt, MD 20771, USA}

%
% ms specific numbers
%

%%%%%%%%%
% STARS %
%%%%%%%%%
\newcommand{\nstarssearched}{65{,}760}
\newcommand{\nlcssearched}{180{,}017}


%%%%%%%%
% CQVS %
%%%%%%%%

% dip-counting pipeline
\newcommand{\nuniqdipflagged}{{368}} % up-to-date; get_lgb_list_count.py

\newcommand{\ncpvsfound}{71}
\newcommand{\ngoods}{54}
\newcommand{\nmaybes}{17}
\newcommand{\ngoodsfieldbanyan}{3}
\newcommand{\nmaybesfieldbanyan}{0}
\newcommand{\ngoodsnotfieldbanyan}{51}
\newcommand{\nmaybesnotfieldbanyan}{17}
\newcommand{\nnotfieldbanyan}{68}


\begin{document}

\title{Corotating Clumps Around Adolescent Low-Mass Stars: Four Years of Complex Quasiperiodic Variables from TESS}

\correspondingauthor{Luke G. Bouma}
\email{luke@astro.caltech.edu}

\received{---}
\revised{---}
\accepted{---}
\shorttitle{Four Years of CQVs} 

\shortauthors{Bouma, Jayaraman, et al.}

%%%%%%%%%%%%%%%%%%%%%%%%%%%%%%%%%%%%%%%%%%
%%%%%%%%%%%%%%%%%%%%%%%%%%%%%%%%%%%%%%%%%%
%%%%%%%%%%%%%%%%%%%%%%%%%%%%%%%%%%%%%%%%%%

% primary authors
\author[0000-0002-0514-5538]{Luke~G.~Bouma}
\altaffiliation{51 Pegasi b Fellow}
\affiliation{\caltech}

\author[0000-0002-7778-3117]{Rahul~Jayaraman}
\affiliation{\mitkavli}

\author{Saul~Rappaport}
\affiliation{\mitkavli}

\author[0000-0001-6381-515X]{Luisa M. Rebull}
\affiliation{\caltech}

\author[0000-0003-4062-0776]{Alexandre David-Uraz}
\affiliation{\howard}
\affiliation{\goddard}

\author{Lynne~A.~Hillenbrand}
\affiliation{\caltech}

\author[0000-0001-7204-6727]{G\'asp\'ar \'A. Bakos}
\affiliation{\princeton}

\author{-----SUGGESTED CONTRIBUTORS, PENDING DISCUSSION, COMMENTS, ANALYSES, ETC-----}
%

\author[0000-0002-3164-9086]{Maximilian N. G\"unther}
\affiliation{\esa}

\author[0000-0003-2058-6662]{George~R.~Ricker}
\affiliation{\mitkavli}

\author[0000-0002-4265-047X]{Joshua N. Winn}
\affiliation{\princeton}


%%%%%%%%%%%%%%%%%%%%%%%%%%%%%%%%%%%%%%%%%%
%%%%%%%%%%%%%%%%%%%%%%%%%%%%%%%%%%%%%%%%%%
%%%%%%%%%%%%%%%%%%%%%%%%%%%%%%%%%%%%%%%%%%

% 250 word max!
\begin{abstract}
  Complex quasiperiodic variables (CQVs) are low-mass
  pre-main-sequence stars with nearly periodic optical modulation.
  The modulation is likely induced by dust or gas clumps in orbit at
  the Keplerian corotation radius.  Here, we report new CQVs
  discovered in TESS short-cadence data collected between July~2018
  and Sep~2022.  Our search of \nstarssearched\ K and M dwarfs with
  $T$$<$16 and $d$$<$150\,pc yielded \ngoods\ high-quality CQVs.  Most
  of these discoveries are new, and they include the brightest
  ($T$$\approx$9.5), closest ($d$$\approx$20\,pc), and oldest
  ($\approx$200\,Myr) examples of this class of object currently
  known.  A few objects are outliers: LP 12-502 for instance showed a
  ``dip complex'' with a period and duration that were fixed over
  1,500 cycles, but in detail, this system exhibited between four and
  eight local minima per cycle, and at times also displayed multiple
  periods simultaneously.  We demonstrate that transient corotating
  material is the most viable explanation for this class of object,
  and present evidence supporting the hypothesis that this material is
  sculpted by stellar magnetic fields dominated by high order
  multipoles.  We expect that our sample will facilitate future
  efforts aimed at connecting CQVs to the broader contexts of star,
  disk, and perhaps even exoplanet evolution. 	
\end{abstract}


\keywords{Weak-line T Tauri stars (1795),
Periodic variable stars (1213),
Circumstellar matter (241),
Star clusters (1567),
Stellar magnetic fields (1610),
Stellar rotation (1629)}


\section{Introduction}
\label{sec:intro}

All pre-main-sequence stars vary in optical brightness, and the origin
of such variability is, in most cases, understood.  Well-explored
sources of optical variability include inhomogeneities on stellar
surfaces such as starspots and faculae
\citep[e.g.][]{2021isma.book.....B}, occultations by gas-rich
circumstellar disks \citep[e.g.][]{2017MNRAS.470..202B}, and, in
geometrically favorable circumstances, eclipses by stars and planets
\citep[e.g.][]{2010exop.book...55W}.  More exotic forms of optical
variability relevant to this work include transiting exocomets
(e.g.~$\beta$~Pic; \citealt{2019A&A...625L..13Z}) and disintegrating
rocky bodies around both M dwarfs (e.g. KOI-2700;
\citealt{2014ApJ...784...40R}) and white dwarfs (e.g. WD~1145;
\citealt{2015Natur.526..546V}).

Data from K2 and TESS have yielded a new class of variable star whose
root cause is only beginning to become clear: complex quasiperiodic
variables (CQVs).  These objects are identified from their optical
light curves, which show nearly periodic troughs that are either sharp
or broad; these troughs are often superposed on smooth spot-like
modulation
\citep{2017AJ....153..152S,2018AJ....155...63S,2019ApJ...876..127Z}.
Some CQVs show up to eight local minima (``dips'') per cycle.  Most
are pre-main-sequence M dwarfs without near-infrared excesses, ages of
$\approx$5-150 million years (Myr), and rotation periods of at most
two days; they are observed to comprise $\approx$1-3\% of M dwarfs
younger than 100 Myr \citep{2016AJ....152..114R,2022AJ....163..144G}.
The dips can be chromatic, with a reddening law plausibly consistent
with dust
\citep{2017PASJ...69L...2O,2020AJ....160...86B,2022AJ....163..144G,2023MNRAS.518.2921K}.
And finally, while the dip shapes can ``jump'' between different
depths and durations over less than one cycle, they more often evolve
gradually, over tens to hundreds of cycles
\citep[e.g.][]{2017AJ....153..152S,2022ApJ...925...75P,2023ApJ...945..114P}.

The CQV light curves cannot be explained by pure starspots.  Although
all young M dwarfs are spotted, this tends to produce flux variations
over timescales of the rotation period, $P_{\rm rot}$,
and its half-harmonic, $0.5\,P_{\rm rot}$.  With fine-tuned
viewing geometries, starspots can produce dip durations as short
as $\approx$$0.2\,P_{\rm rot}$, and the dip amplitudes in such
geometries are small
due to limb-darkening (see \citealt{2017AJ....153..152S},
Figures 37-41).  The observed dips occur over durations as
short as $0.05\,P_{\rm rot}$ with characteristic depths of a few
percent; a ``starspot-only'' scenario can be discarded for any object
with dips that are sufficiently large in depth and short in duration
\citep{2017AJ....153..152S,2021MNRAS.500.1366K}.   The dip durations and
amplitudes instead require
sharp geometries with material extrinsic to the stellar surface
\citep[e.g.][]{2017AJ....153..152S,2022AJ....163..144G}.  Since some CQVs require
such geometries, and the
CQVs probably all have a single explanation, we therefore
discard the ``spot-only'' scenario.

A few competing explanations for what causes the complex quasiperiodic
variability are shown in Figure~\ref{fig:f1}.  The ``clump'' scenario
invokes opaque dust clumps periodically transiting the star near the
Keplerian corotation radius, $R_{\rm cr} = (GM/\Omega^2)^{1/3}$
\citep{2017AJ....153..152S,2023MNRAS.518.4734S}.  The ``prominence''
scenario invokes long-lived condensations of cool and dense
marginally-ionized gas, embedded within the hotter corona, that would
be centrifugally supported to also be near corotation
\citep{1989MNRAS.238..657C,2019MNRAS.482.2853J,2022MNRAS.514.5465W}.
Such structures are analogous to quiescent prominences and filaments
seen in the solar corona \citep[see e.g.][]{2015ASSL..415.....V},
though at much larger relative distances from the stellar surface.  A
final possibility is one of a ``screen'', in which the inner wall of a
quiescent circumstellar disk blocks a portion of the stellar surface
to produce sudden dips whenever spots come into view
\citep{2019ApJ...876..127Z}.  

The dust clump and prominence hypotheses seem most plausible.  They
are qualitatively similar, except that one invokes opacity from dust,
while the other invokes opacity from gas, likely bound-free
transitions in hydrogen or perhaps a molecular opacity.  The screen
scenario seems inconsistent with the sharpness of the observed dips,
the lack of infrared excess, and the observed lifetime of CQVs
extending an order of magnitude longer than the $\approx$$10^7$ year
timescale often quoted for primordial disk dispersal.  However,
unambiguous evidence in support of any one scenario has yet to be
acquired.  Such evidence might include a spectroscopic detection of
silicate 10\,$\mu$m dust absorption during a dip, or perhaps detection
of transient Balmer-line excesses as a function of cycle phase,
similar to observations made in systems such as AB~Dor
\citep[see][]{1999ASPC..158..146C} or PTFO 8-8695
\citep{2016ApJ...830...15J}.

CQVs remain mysterious because they have been both hard to discover
and hard to characterize.   Discoverability is tied to rarity: CQVs
comprise $\approx$1\% of the youngest $\approx$1\% of M dwarfs
\citep{2018AJ....155..196R}.  Out of the millions of stars monitored
by K2 and TESS, about 50 CQVs have been reported to date
\citep{2016AJ....152..114R,2017AJ....153..152S,2018AJ....155...63S,2019ApJ...876..127Z,2020AJ....160...86B,2022AJ....163..144G,2023ApJ...945..114P}.
The known CQVs are correspondingly faint; the initial K2 discoveries
\citep{2016AJ....152..114R,2017AJ....153..152S} were M2-M6 dwarfs at
distances $\gtrsim$100\,pc, yielding optical brightnesses of
$V$$\approx$15.5 to $V$$>$20.  This renders time-series spectroscopy
at high resolution out of reach for current facilities, despite its
potential utility in ruling out some of these models.

In this work, we aim to find bright and nearby CQVs, since these
objects will be the most amenable to detailed photometric and
spectroscopic analyses.  To do this, we use 120-second cadence data
acquired by TESS between 2018 July and 2022 September (Sectors 1-55;
Cycles 1-4).  We present our search methods in
Section~\ref{sec:methods}, and the properties of the resulting CQV
catalog in Section~\ref{sec:results}.  The evolution of many CQVs over
a two-year baseline is described in Section~\ref{sec:evoln}, including
a deep-dive into LP 12-502.  We discuss a few implications in
Section~\ref{sec:discussion}, and conclude in
Section~\ref{sec:conclusion}.

A point on nomenclature.  CQVs have been called ``transient flux
dips'', ``persistent flux dips'', ``scallop shells'', ``batwings'',
\citep{2017AJ....153..152S} ``complex rotators'',
\citep{2019ApJ...876..127Z,2022AJ....163..144G,2023ApJ...945..114P},
and ``complex periodic variables'' \citep{2023MNRAS.518.2921K}.  The
CQVs should not be conflated with ``dippers'', which are classical T
Tauri stars with infrared excesses, and which show large-amplitude
variability linked to obscuring inner disk structures and accretion
hot spots \citep{2014AJ....147...82C,2021ApJ...908...16R}.  While a
few similarities between CQVs and dippers do exist (see
Sections~\ref{subsec:irexcess} and~\ref{subsec:discdippers}), their
phenomenology and stellar properties are quite different.  At the risk
of introducing yet another standard, we hope to use a nomenclature
that reflects how, when observed over timescales of more than tens of
cycles, CQVs are almost but not exactly periodic.  They are
quasiperiodic.  While the three-type classification scheme proposed by
\citet{2017AJ....153..152S} may indeed provide some helpful visual
distinctions amongst the CQVs, it seems likely that they are all
explained by a single underlying phenomenon, and so we opt to refer to
them by a single empirically descriptive name.  We also considered
names that focused on the presence of transient corotating clumps, or
more generally on circumstellar plasma tori; for the time being,
``CQV'' is agnostic to the true underlying phenomenon.

\begin{figure*}[!t]
	\begin{center}
		\subfloat{
			\includegraphics[width=0.8\textwidth]{f1a.pdf}
		}
		
		\vspace{-0.6cm}
		\subfloat{
			\includegraphics[width=0.8\textwidth]{f1b.pdf}
		}
	\end{center}
		\vspace{-0.3cm}
	\caption{
		{\bf Complex quasiperiodic variables (CQVs)}:
    {\it Top:} Phase-folded TESS light curves of three CQVs.  Each is
    stacked over one month.  Gray circles are raw 2-minute data; black
    circles bin to 300 points per cycle.  Periods in hours are in the
    bottom right of each panel.  Left-to-right, the objects are LP
    12-502 (TIC 402980664; Sector~19), TIC 94088626 (Sector 10), and
    TIC 425933644 (Sector~28).
    {\it Bottom:} Cartoon explanations for the phenomenon.  The dust clump
    scenario (lower left) and prominence scenario (lower center) both
    invoke centrifugally-supported material at the corotation radius.
    We disfavor the screen scenario (see Section~\ref{sec:intro}).
	}
	\label{fig:f1}
\end{figure*}



\section{Methods}
\label{sec:methods}

\subsection{Stellar selection function}
\label{subsec:selectionfn}

To construct our sample, we analyzed the short-cadence data acquired
by TESS between 2018 July 25 and 2022 September 1 (Sectors 1-55).
Specifically, we used the 120-second cadence light curves produced by
the Science Processing and Operations Center at NASA Ames
\citep{2016SPIE.9913E..3EJ}.  While the TESS data products from these
sectors also included full frame images with cadences of 600 and 1800
seconds, we limited our scope in this study for the sake of simplicity
in data handling and some degree of uniformity in the properties of
the data itself.  In exchange, we sacrificed both completeness and
homogeneity of the selection function.  While TESS cumulatively
observed $\approx$90\% of the sky for at least one lunar month between
2018 July and 2022 September, the 120-second cadence data were
collected for only a subset of observable stars during this time due
to telemetric constraints \citep[see][]{2021PASP..133i5002F}.

To assess the completeness of the resulting 120-second cadence data
that is the basis of this study, we cross-matched TIC8.2
\citep{2019AJ....158..138S,2021arXiv210804778P} against the Gaia DR2 point-source catalog
\citep{2018A&A...616A...1G}.  We opted for Gaia DR2 rather than DR3
because the base catalog for TIC8 was Gaia DR2, which facilitated a
one-to-one crossmatch using the Gaia source identifiers.  This
exercise showed that for $T$$<$16 M dwarfs, the TESS 2-minute data are
$\approx$50\% complete at $\approx$50\,pc.  At $<$20\,pc,
$\gtrsim$80\% of the $T$$<$16 M dwarfs have at least one sector of
short-cadence data; at $>$100\,pc, $\lesssim$10\% of such M dwarfs
have at least one sector of short-cadence data.  Armed with this
understanding, we then used our cross-match between Gaia DR2 and TIC8
to select our stars of interest, which we defined as stars with
120-second cadence TESS light curves satisfying these four conditions:
\begin{align}
  T &< 16 \quad&(\mathrm{Bright\ for\ TESS})\\
  \bprp &> 1.5 \quad&(\mathrm{Red\ stars\ only})\\
  M_{\rm G} &> 4 \quad&(\mathrm{Dwarf\ stars\ only})\\
  d &< 150\,{\rm pc} \quad&(\mathrm{Close\ stars\ only}).
\end{align}
Here, $M_{\rm G} = G + 5\log(\varpi_{\rm as}) + 5$ is the Gaia
$G$-band absolute magnitude, $\varpi_{\rm as}$ is the parallax in
units of arcseconds, and $d$ is a geometric distance defined by
inverting the parallax and ignoring any zero-point correction.  The
target sample therefore includes \nstarssearched\ M dwarfs and late-K
dwarfs, down to $T$$<$16 and out to $d$$<$150\,pc.

\subsection{CQV discovery}
\label{subsec:discoverymethods}

Previous methods for finding CQVs have included visually examining
stars known to be in young clusters
\citep{2016AJ....152..114R,2017AJ....153..152S}, and automatically
flagging rapid rotators with a large number of strong Fourier
harmonics \citep{2019ApJ...876..127Z}.  The latter approach still
requires visual vetting, since ``stars with many Fourier harmonics''
is a designation that includes objects such as eclipsing binaries or
multiple stars blended into a single photometric aperture.  In this
work, we implemented a new search approach based on counting the
number of sharp local minima in phase-folded light curves, while also
using the previously tested Fourier approach.  We applied these two
search techniques independently.   


\subsubsection{Counting dips}
\label{subsec:counting}

The dip counting technique aims to count sharp local minima in
phase-folded light curves.  CQVs will preferably have at least three
such minima in order to be distinct from false positives such as
synchronized and spotted binaries (``RS CVn'' stars). 

For our dip-counting pipeline, we began with the {\tt PDC\_SAP} flux
for each sector, removed non-zero quality flags, and normalized the
light curve to one by dividing out its median value.  We then
flattened the light curve using a 5-day sliding median filter, as
implemented in \texttt{wotan} \citep{2019AJ....158..143H}.  On the
resulting cleaned and flattened light curve, we ran a periodogram
search, opting for the \citet{1978ApJ...224..953S} phase dispersion
minimization (PDM) algorithm implemented in \texttt{astrobase}
\citep{2021zndo...1011188B} due to its shape agnosticism.  If a period
$P$ below 2 days was identified, we reran the periodogram at a finer
grid to improve the accuracy of the period determination.

Once a star's period was identified, we binned the phased light curve
to 100 points per cycle.  To separate ``sharp'' local minima from
smooth but double-dipped spot variability, we then iteratively fit
robust penalized splines to the wrapped phase-folded light curve,
excluding points more than two standard deviations away from the local
continuum \citep{2019AJ....158..143H}.  The maximum number of
equidistant spline knots per cycle is the parameter in this framework
that controlled the meaning of ``sharp'' --- we allowed at most 10
such knots per cycle, though for most stars fewer knots were preferred
based on cross-validation using an $\ell^2$-norm penalty.  An example
fit is shown in panel {\it (e)} of Figure~\ref{fig:vet}.

We then identified local minima in the resulting residual light curve
using the SciPy \texttt{find\_peaks} utility
\citep{2020NatMe..17..261V}, which is based on comparing adjacent
values in an array.  For a peak to be flagged as significant, we
required it to have a width of at least $0.02\,P$, and a height of at
least twice the point-to-point RMS.  This latter quantity is defined
as the 68$^{\rm th}$ percentile of the distribution of the residuals
from the median value of $\delta f_i \equiv f_i - f_{i+1}$, where $f$
is the flux and $i$ is an index over time.

To identify local minima near the edges of the phased light curve,
which usually would cover phases $\phi \in [ 0,1 ]$, we performed the
entire procedure over a phase-folded light curve spanning $\phi \in
[-1,2 ]$, by duplicating and concatenating the ordinary phase-folded
light curve.  The free parameters we adopted throughout this analysis
procedure, for instance the maximum number of spline knots per cycle,
and how large and wide of a local minimum to consider a ``true dip'',
were chosen during testing based on their ability to correctly
re-identify a large fraction ($>$90\%) of known CQVs, while also being
able to consistently reject common false positives such as rapidly
rotating spot-induced variability and typical eclipsing binaries.

Overall, for a star to proceed to manual examination, we required that
it have a peak PDM period below two days, and that it exhibited at
least three sharp local minima (as algorithmically reported) in at
least one observed TESS sector.



\subsubsection{Fourier analysis}
\label{subsec:fourier}

We performed an independent search for strongly periodic phenomena
using a Fourier-based approach, following \citet{2019ApJ...876..127Z}
and \citet[][their Section~1.3]{2023MNRAS.524.4220P}.  Starting with
the {\tt PDC\_SAP} light curves, we normalized each light curve, and
then re-binned it into equal width 120-second bins to account for the
uneven spacing in the TESS data, as well as the data gap caused by
satellite downlink during each sector.  We then padded the data to
ensure that the light curve had a length that was a power of two, as
described by \citeauthor{2019ApJ...876..127Z}.  Taking the Fourier
transform of the corrected, padded light curve using {\tt
numpy.fft.fft}, we then searched for peaks with a significance of over
12-$\sigma$ within a set of 500 frequency bins. 

If a peak of such significance was found, we generated a ``summary
sheet'' with information about the star, its full and folded light
curves, Fourier transform, potential contaminating stars, and
information about these contaminating stars.  In the sectors for which
we performed this analysis, we found that ``summary'' sheets were
generated for $\approx$10\% of the 20\,000 120-second targets in a
given sector.  We then manually reviewed each summary sheet, and
visually labelled each star based on its morphology (including e.g.,
eclipsing binary, CQV, RS CVn, or cataclysmic variable).  CQVs that
were observed in multiple sectors had their light curves stitched
together and analyzed for a more accurate period determination.


\subsubsection{Manual vetting}

\begin{figure*}[!tp]
	\begin{center}
		\centering
		\includegraphics[width=0.96\textwidth]{f2.pdf}
		\vspace{-0.3cm}
		\caption{
			{\bf Validation plots used to  label CQVs}.  The complete figure
			set, with one image per sector for each of the \nallcands\
      objects in Table~\ref{tab:thetable}
			is available in the online journal. {\bf For internal collaboration review:
				\url{https://www.dropbox.com/sh/khtwk5a6z0zgrif/AABRG21Ire7VS4BUnaD1hOR8a?dl=0}}.
			Panels are as follows.
			{\it a)}: Phase-folded light curve; gray points are raw 2-minute
			data and black points are binned to 200 points per cycle.
      The adopted period is given in the lower-right corner.
			{\it b)}: Phase-dispersion minimization (PDM) periodogram.
			Dotted lines show up to the 10$^{\rm th}$ harmonic and
			subharmonic.
			{\it c)}: DSS finder chart, with 1- and 2-TESS pixel radius
			circles displayed for scale.
			{\it d)}: Cleaned light curve, binned to 20-minute cadence, in
			Barycentric TESS Julian Date (BTJD).
			{\it e)}: Phase-folded light curve, binned to 100 points per
			cycle.  The gray line denotes the automated spline-fit to the
			wrapped phase-folded light curve, and small gray triangles
			denote automatically identified local minima.
			{\it f)}: Phase-folded light curve at twice the peak period.
			{\it g)}: Phase-folded light curve at half the peak period.
			{\it h)}: Phase-folded time-series within the ``background''
			aperture defined in the SPOC light curves.
			{\it i)}: Phase-folded flux-weighted centroid in the column
			direction.
			{\it j)}: Phase-folded flux-weighted centroid in the row
			direction.
			{\it k)}: Gaia DR2 color--absolute magnitude diagram.     
			The gray background denotes stars within 100\,pc.
			{\it l)}: Information from Gaia DR2, TIC8, and the automated
			dip-counting search pipeline.  ``Neighbors'', abbreviated
			``nbhr'', are listed within apparent distances of 2 TESS pixels
			if $\Delta T$$<$2.5.
			{\it m)}: BANYAN-$\Sigma$ v1.2 association probabilities, calculated
			using positions, proper motions, and the parallax.
		}
		\label{fig:vet}
	\end{center}
\end{figure*}

We assessed whether the objects found using the dip-counting
(Section~\ref{subsec:counting}) and Fourier
(Section~\ref{subsec:fourier}) techniques were consistent with
expectations for CQVs by assembling the data shown in
Figure~\ref{fig:vet}.  We labeled a star as a ``good'' CQV if at least
one TESS sector showed what we viewed as the unambiguous signatures of
the class (short period; at least three dips or else otherwise
oddly-shaped dips; relative stability over a timescale of 30\,days).
We independently noted stars that we thought could be CQVs, but that
were more ambiguous.

Broadly speaking, the most common false positives for both the Fourier
and dip-counting techniques were eclipsing binaries, spot-induced
variability from rapid rotators, and variability from neighboring,
off-target stars.  Typical false positive rates from our dip-counting
pipeline were 5:1, with \nuniqdipflagged\ unique stars flagged, and
about 20\% being labeled either ``good'' or ``possible'' CQVs.  The
Fourier analysis was not amenable to calculating a similar
false-positive rate, because it was implemented as a general
variability search; its results are also being used for analyses on
topics other than CQVs.


\subsection{Stellar properties}
\label{subsec:starprops}

\paragraph{Ages}
We estimated the stellar ages by making probabilistic spatial and
kinematic associations between the CQVs and known clusters in the
solar neighborhood.  For most stars in our sample, we did this using
BANYAN\,$\Sigma$
\citep{2018ApJ...856...23G}.\footnote{\url{https://github.com/jgagneastro/banyan_sigma},
git commit \texttt{394b486}} This algorithm calculates the probability
that a given star belongs to one of 27 young clusters (or
``assocations'') within 150\,pc of the Sun, by modeling the clusters
as multivariate Gaussians in 3-D position and 3-D velocity space.  We
used the Gaia DR2 sky positions, proper motions, and distances to
calculate the membership probabilities.  BANYAN\,$\Sigma$ in turn
analytically marginalizes over the radial velocity dimension.  The
probabilities returned by this procedure are qualitatively useful, but
should be assessed with caution due to the non-Gaussian nature of most
groups within the solar neighborhood \citep[see
e.g.][Figure~10]{2021ApJ...917...23K}.

For a few cases where BANYAN\,$\Sigma$ yielded ambiguous results, we
consulted the meta-catalog of young, age-dated, and age-dateable stars
within from \citet{2022AJ....163..121B}, and also searched the local
volume around each star for co-moving
companions.\footnote{\url{https://github.com/adamkraus/Comove}, git
commit \texttt{278b372}}


\paragraph{Effective temperatures, radii, and masses}

We determined the stellar effective temperature and radii by fitting
the broadband spectral energy distributions (SEDs); we then estimated
the masses by interpolating against the sizes, temperatures, and ages
of the PARSEC v1.2S models
\citep{2012MNRAS.427..127B,2014MNRAS.444.2525C}.

For the SED fitting, we used \texttt{astroARIADNE}
\citep{2022MNRAS.513.2719V}.  We adopted the BT-Settl stellar
atmosphere models \citep{Allard2012} assuming the
\citet{2009ARA&A..47..481A} solar abundances, and the
\citet{2006MNRAS.368.1087B} water line lists.  The broadband
magnitudes we considered included $GG_{\rm BP}G_{\rm RP}$ from Gaia
DR2, $Vgri$ from APASS, $JHK_{\rm S}$ from 2MASS, SDSS $riz$, and the
WISE $W1$ and $W2$ passbands.  We omitted UV flux measurements from
our SED fit to avoid any possible bias induced by chromospheric UV
excess; we similarly omitted WISE bands $W3$ and $W4$, but due to
reliability concerns.  \texttt{astroARIADNE} compares the measured
broadband flux measurements against pre-computed model grids, and by
default fits for six parameters: $\{ T_{\rm eff}, R_\star, A_{\rm V},
\log g, [{\rm Fe/H}], d \}$.  The distance  prior is drawn from
\citet{2021AJ....161..147B}.  The surface gravity and metallicity are
generally unconstrained.  And finally, given our particular use-case,
we assumed the following priors for the temperature, stellar size, and
extinction:
\begin{align}
  T_{\rm eff} / {\rm K}    &\sim \mathcal{N}(3000, 1000), \\
  R_\star / R_\odot  &\sim \mathcal{T}_{\rm N}(0.5, 0.3, 0.1, 1.5), \\
  A_{\rm V} / {\rm mag}    &\sim \mathcal{U}(0, 0.2),
\end{align}
for $\mathcal{N}$ the Gaussian and $\mathcal{U}$ the uniform
distributions, and $\mathcal{T}(\mu, \sigma, a, b)$ a truncated normal
distribution with mean $\mu$, standard deviation $\sigma$, and lower
and upper bounds $a$ and $b$.  We validated our chosen upper bound on
$A_{\rm V}$ using a 2MASS color-color diagram.  Finally, using
\texttt{Dynesty} \citep{2020MNRAS.493.3132S}, we sampled
the posterior probability assuming the default Gaussian likelihood,
and set a stopping threshold of ${\rm d}\log \mathcal{Z} < 0.01$,
where $\mathcal{Z}$ denotes the evidence.

With the effective temperatures and stellar radii from the SED fit, we
then estimated the stellar masses by interpolating against the PARSEC
isochrones \citep[v1.2S][]{2014MNRAS.444.2525C}.  The need for models
that incorporate some form of correction for young, active M dwarfs is
well-documented
\citep[e.g.][]{2012ApJ...756...47S,2015ApJ...804..146D,2016A&A...593A..99F,2020ApJ...891...29S}.
Plausible explanations for anomalous M dwarf colors and sizes relative
to model predictions include starspot coverage
\citep[e.g.][]{2017ApJ...836..200G}, and potentially incomplete line
lists \citep[e.g.][]{2013A&A...556A..15R}.  In the PARSEC models,
\citet{2014MNRAS.444.2525C} performed an empirical correction to the
temperature--opacity relation drawn from the BT-Settl model
atmospheres, in order to match observed masses and radii of younq
eclipsing binaries.  This is sufficient for our goal of estimating
stellar masses.  Given our observed $\{ \tilde{T}_{\rm eff},
\tilde{M}_\star, \tilde{t} \}$, and approximating their uncertainties
as Gaussian $\sigma_{\tilde{T}_{\rm eff}}$, $\sigma_{\tilde{M}_\star}$
and $\sigma_{\tilde{t}}$, we evaluate a distance $d$ between our
observations and any model PARSEC grid-point $\{ T_{\rm eff}, M_\star,
t \}$ as
\begin{equation}
  d^2 = 
  \left( \frac{\tilde{T}_{\rm eff} - T_{\rm eff}}{\sigma_{\tilde{T}_{\rm eff}}} \right)^2
  +
  \left( \frac{\tilde{M}_{\star} - M_{\star}}{\sigma_{\tilde{M}_{\star}}} \right)^2
  +
  \left( \frac{\tilde{t} - t}{\sigma_{\tilde{t}}} \right)^2,
\end{equation}
in order to assign equal importance to each dimension.  The preferred
model mass is then one that minimizes this distance, and is quoted in
Table~\ref{tab:thetable}.
{\bf lgb todo: real interpolation, rather than nearest neighbor}



\begin{figure*}[!tp]
	\begin{center}
		\centering
		\includegraphics[width=0.98\textwidth]{f3.pdf}
    \vspace{-0.3cm}
		\caption{
      {\bf CQVs found in the TESS 2-minute data.}
      Phased TESS light curves over one month are shown for \ngoods\
      CQVs in the high quality sample.  Gray are raw 2-minute data;
      black bins to 300 points per cycle.  Objects are ordered such
      that sources with the most TESS data available are on top (see
      Section~\ref{sec:catalog}).  
      Each panel is labeled by
      the TIC identifier, the TESS sector number, the period in
      hours, and the three-bit binarity flag from Table~\ref{tab:thetable},
      which denotes Gaia DR3 \texttt{radial\_velocity\_error} outliers
      (bit 1), Gaia DR3 \texttt{ruwe} outliers (bit 2), and stars with
      secondary TESS periods (bit 3). 
		}
		\label{fig:cqvs}
	\end{center}
\end{figure*}



\section{Results}
\label{sec:results}

\subsection{CQV catalog}
\label{sec:catalog}

Table~\ref{tab:thetable} lists the \nallcands\ objects identified by
our search.  The ``high-quality sample'' includes \ngoods\ objects,
which demonstrated what we viewed as unambiguous characteristics of
the CQV phenomenon in at least one TESS sector.  The classification of
\nmaybes\ CQV candidates was ambiguous, and the \ndebunked\ remaining
objects were false positives that we discuss below.  The
\texttt{quality} column in the table divides the three classes;
additional data from TESS or other instruments could help resolve the
classification of the ambiguous cases.   Of the \ncqvsnodebunked\ CQVs
and candidate CQVs, \nbothdipfourier\ were found using both the
dip-counting and Fourier techniques, \nyesdipnofourier\ were found
using only the dip-counting technique, and \nyesfouriernodip\ were
found using only the Fourier technique.  In the following, we will
restrict our discussion to the high-quality sample, irrespective of
discovery method.

The mosaic in Figure~\ref{fig:cqvs} shows phased light curves for the
\ngoods\ CQVs.  The objects are sorted first in order of the number of
TESS 120-second cadence sectors in which they clearly demonstrated the
CQV phenomenon, and secondarily by descending brightness.  The top
five objects by this metric are TIC~300651846 ($T$$=$13.5, 12
sectors); TIC~402980664 ($T$$=$11.1, 7 sectors); TIC~89463560
($T$$=$13.5, 5 sectors); TIC~363963079 ($T$$=$12.9, 5 sectors); and
TIC~294328887 ($T$$=$14.2, 4 sectors).  The brightest five CQVs span
9.3$<$$T$$<$11.1; the faintest five span 14.5$<$$T$$<$15.0.  The
fastest five have periods spanning 3.6\,hr$<$$P$$<$6.2\,hr, and the
slowest five span 27\,hr$<$$P$$<$38\,hr.

In terms of the light curve shapes, Figure~\ref{fig:cqvs} shows a
broad range of variability, with anywhere from two to eight local
minima per cycle.  Some stars show relatively ordinary modulation
during one portion of the phased light curve, and highly structured
modulation in the remainder of the cycle (e.g.  TIC~206544316,
TIC~224283342, TIC~402980664).  Others show structured modulation over
the entire span of a cycle (e.g. TIC~2234692, TIC~425933644,
TIC~142173958).  Others show some mix between these two modes.

A small number of objects at first glance seem reminiscent of
eclipsing binaries, such as TIC~193831684, TIC~59836633, or
TIC~5714469.  We believe these cases are unlikely to be eclipsing
binaries due to the additional coherent peaks and troughs in the light
curves, which are distinct from any binary phenomena of which we are
aware.



\subsection{Ages}

Of our \ncqvsnodebunked\ confirmed and candidate CQVs,
\nnotfieldbanyan\ were associated with a nearby moving group or open
cluster, primarily using BANYAN\,$\Sigma$.\footnote{Two of the
\nnotfieldbanyan\ memberships were made with low confidence and are
flagged in Table~\ref{tab:thetable}.  One membership, of TIC~397791443
in IC\,2602, was made manually from a literature search
\citep[e.g.][]{2020A&A...633A..99C}.}  The relevant groups are listed
in Table~\ref{tab:thetable}; their ages span $\approx$5-200\,Myr.  The
most prodigious groups were Sco-Cen, Tuc-Hor, and Columba.  Six CQVs
were also identified in the Argus association
\citep{2019ApJ...870...27Z}, which serves as an indirect line of
evidence supporting the reality and youth of that group.  The yield in
Sco-Cen is not surprising since Sco-Cen contains the majority of
pre-main-sequence stars in the solar neighborhood, and many of its
stars were selected for TESS 120-second cadence observations by
various guest investigators.  Given the $\lesssim$$10\%$ completeness
of TESS beyond 100\,pc, there may be many more CQVs in Sco-Cen that
remain to be discovered.  

For the two stars for which BANYAN\,$\Sigma$ did not find any
association, we were not able to confidently associate either star
with any young groups.  Both do seem to show the CQV signal over
multiple TESS sectors, and both are photometrically elevated relative
to the main sequence.  Both were also noted by
\citet{2021ApJ...917...23K} as being in the ``diffuse'' population of
$<$50\,Myr stars near the Sun.

Our search confirms that the CQV phenomenon persists for at least
$\approx$150\,Myr.  Table~\ref{tab:thetable} includes three
$\approx$150\,Myr CQVs in AB~Dor \citep{2015MNRAS.454..593B}, a
$\approx$112\,Myr old Pleiades CQV \citep{2015ApJ...813..108D}, and a
similarly-aged Psc-Eri member \citep{2020A&A...639A..64R}.  To our
knowledge, TIC~332517282 in AB~Dor was the previous record-holder for
the oldest-known CQV \citep{2019ApJ...876..127Z,2022AJ....163..144G};
at least one unambiguous CQV (EPIC~211070495) and a few other
candidates were also previously known in the Pleiades
\citep{2016AJ....152..114R}.  

The upper age limit for CQVs may even pass 200\,Myr, based on the
candidate membership of TIC~294328887 in the Carina Near moving group
\citep{2006ApJ...649L.115Z}.  This group's $200 \pm 50$\,Myr age is
based on the lithium sequence of its G-dwarfs
\citep{2006ApJ...649L.115Z}, which shows a coeval population of stars
older than the Pleiades but younger than the 400\,Myr Ursa Major
moving group.  The formal BANYAN-$\Sigma$ membership probability is
however somewhat low, perhaps due to the missing radial velocity.
This lack of information could be resolved by acquiring even a
medium-resolution spectrum.  An independent assessment of the group's
kinematics using Gaia data, and its rotation sequence using TESS,
could lend further credence in such an analysis.


\subsection{Infrared excesses}
\label{subsec:irexcess}

Most CQVs in our catalog did not show infrared excesses in the
$W1$-$W4$ bands, which is typical for this class of object
\citep{2017AJ....153..152S}.  Visually inspecting the SEDs of our
\nallcands\ star sample and the WISE images available through IRSA, we
labeled two objects as having reliable infrared excesses (both $W3$
and $W4$ at $>$3$\sigma$ above the photospheric prediction).

The two sources exhibiting an IR excess were TIC~193136669 (TWA~34)
and TIC~57830249 (TWA~33).  Both are in the TW Hydrae association
($\approx$10\,Myr), and have relatively long periods of 38\,hr and
44\,hr respectively.  In our initial labeling, we labeled both as
``ambiguous'' CQVs because the dips in their light curves did not show
the rigid periodicity typical of CQVs;  their periods were also
relatively long.  However, inspection of further sectors clarified
that both sources are dippers (see online plots in
Figure~\ref{fig:vet}).  Independently, TIC 193136669 has a cold
molecular disk based on observed 1.3\,mm continuum emission and
resolved Keplerian $^{12}{\rm CO}(2-1)$ emission
\citep{2015A&A...582L...5R}.  It was also labeled a dipper by
\citet{2022ApJS..263...14C}; we agree with their designation, and
label it an ``impostor'' CQV in Table~\ref{tab:thetable}.  TIC
57830249 (TWA~33) also has previously detected 1.3\,mm continuum
emission \citep{2015A&A...582L...5R}, suggestive of cold dust grains
being present.  It is also a dipper.  Section~\ref{subsec:discdippers}
highlights plausible evolutionary connections between CQVs and dippers
in light of these ``misclassifications'' .

% Our three ``possible'' infrared excesses were TIC~405910546,
% 289840926, and 244161191.  After a literature search, we concluded
% that none have clear evidence for the presence of a disk.
% TIC~405910546, in LCC, shows a unique TESS light curve, reminiscent of
% a $P$=38\,hr singly-eclipsing binary, except with additional
% substructure during each eclipse that resembles the CQV phenomenon
% more than any other variability of which we are aware.  TIC~289840926
% ($\beta$~Pic moving group, $P$=4.8\,hr), shows what we believe is a
% clear CQV signal, but has no definitive evidence for a large, dusty
% disk.  TIC~244161191 (TOI-278), in Columba, also has no definitive
% evidence for a large disk.  It is however ``multi-periodic''---in
% addition to the 7.17\,hr CQV signal, this source shows a superposed
% 8.39\,hr sinusoidal signal, probably from an unresolved neighboring
% star or binary companion.



\begin{figure*}[!t]
	\begin{center}
		\centering
		\includegraphics[width=0.95\textwidth]{f8.pdf}
		\vspace{-0.1cm}
		\caption{
			{\bf Properties of CQVs identified by our search}.
      CQVs are mostly pre-main-sequence M dwarfs, younger than
      $\approx$150 Myr, with rotation periods faster than
      $\approx$1\,day.  The \ngoods\ bona fide CQVs in
      Table~\ref{tab:thetable} are the dark blue circles; \nmaybes\
      ambiguous CQV candidates are light blue circles.  Unresolved
      binaries are flagged either if the Gaia DR3 radial velocity
      scatter exceeded 20\,\kms, or if Gaia ${\rm RUWE}_{\rm DR3}$$>$2
      and multiple photometric signals were present in the TESS light
      curve.  The top panels show the \nstarssearched\ target stars
      with 120-second cadence TESS data as the shaded gray background;
      darker regions correspond to a larger relative number of
      searched stars.  The lower-left panel compares the
      rotation--color distribution of CQVs against the rotation
      periods of K and M dwarfs in the Pleiades from
      \citet{2016AJ....152..114R}.  The lower-middle panel plots the
      derived corotation radii $R_{\rm cr} = (GM/\Omega^2)^{1/3}$ in
      units of stellar radii against the measured CQV periods, in
      units of hours.  Ages in the final panel are known from cluster
			membership.
		}
		\label{fig:catalogscatter}
	\end{center}
\end{figure*}



\subsection{Binarity}
\label{subsec:binarity}

The main types of binaries of interest in this work were those that
were unresolved, because they could confuse our understanding of CQVs.
For instance, unresolved binaries might produce multiple photometric
signals and hinder our ability to correctly identify the star hosting
the CQV signal.  Unresolved binaries could also bias photometric magnitude
and color measurements, which would affect our stellar parameter
estimates.  To attempt to identify binaries, we considered the
following lines of information.

{\it Radial velocity scatter}---We examined diagrams of the Gaia DR3
``radial velocity error'' as a function of stellar color for all
\ncqvsnodebunked\ CQVs and candidate CQVs.  Since this quantity
represents the standard deviation of the non-published Gaia RV time
series, outliers can suggest single-lined spectroscopic binarity.
These plots showed two clusters of stars, at $\lesssim$10\,\kms and
20-25\,\kms.  We there adopted a threshold of $20$\,km\,s$^{-1}$ to
flag possible single-lined spectroscopic binaries, which selected
\nrvscatterflag\ stars: TIC~405910546, TIC~224283342, and
TIC~280945693.

{\it RUWE}---We examined plots of Gaia DR3 RUWE as a function of
color.\footnote{For an explanation of the renormalized unit weight
error (RUWE), see the GAIA DPAC technical note
\url{http://www.rssd.esa.int/doc_fetch.php?  id=3757412}.}  Elevated
RUWEs imply excess noise relative to a single-source astrometric
model.  This can be caused by marginally resolved point sources
skewing the single-transit centroid measurements,  intrinsic
photometric variability, or intrinsic astrometric motion.  Based on
this exercise, we adopted a threshold of RUWE$_{\rm DR3}$ $>$ 2 to
flag sources with excess astrometric noise.  This threshold was met
for \ngoodhighruwe/\ngoods\ high-quality CQVs;
\nmaybehighruwe/\nmaybes\ of the ambiguous ones had this
characteristic.  There is some subjectivity in where to set the
threshold, since the RUWE distribution has an extended tail
\citep[e.g.][]{2022MNRAS.513.5270P}.  If we had instead required
RUWE$_{\rm DR3}$ $>$ 1.4, \ngoodweakruwe/\ngoods\ high-quality CQVs
and \nmaybeweakruwe/\nmaybes\ of the ambiguous sample would have been
flagged.

{\it Gaia DR3 non-single stars}---Gaia DR3 included a {\tt
non\_single\_star} column that flagged known or suspected eclipsing,
astrometric, and spectroscopic binaries.  None of the stars in our CQV
sample were identified as possible binaries in this column.

{\it Multiple periodic TESS signals}---During our visual analysis of
the TESS light curves and PDM periodograms, we flagged sources with
beating light curves, and with PDM periodograms that showed multiple
periods.  We also attempted to disentangle the two signals by
subtracting the mean CQV signal over each sector, and then repeating
the phase-dispersion minization analysis.  The results of this effort
are summarized in the secondary periods, $P_{\rm sec}$, listed in
Table~\ref{tab:thetable}.  This process yielded
\ngoodmultperiodflag/\ngoods\ high-quality CQVs with secondary
periods; \nmaybemultperiodflag/\nmaybes\ of the ambiguous sample met
the same criterion.  Of the \ngoodhighruwe\ good CQVs with RUWE$_{\rm
DR3}$ $>$ 2, \ngoodruweandmultperiod\ also showed secondary periods in
the TESS light curves.  Considering the weaker threshold of RUWE$_{\rm
DR3}$ $>$ 1.4, \ngoodweakruweandmultperiod/\ngoodweakruwe\ such CQVs
showed secondary TESS periods.

Table~\ref{tab:thetable} summarizes each of the above sources of
information into a single bitwise column.  We discuss possible
connections between binarity and the CQV phenomenon in
Section~\ref{subsec:discbinary}.


\section{CQV evolution}
\label{sec:evoln}

\begin{figure*}[!tp]
	\begin{center}
		\centering
		\includegraphics[width=\textwidth]{f4.pdf}
		\vspace{-0.6cm}
		\caption{
			{\bf CQVs keep their periods but change their shapes.}
			Out of the \ngoods\ CQVs in Figure~\ref{fig:cqvs}, 32 had
			120-second cadence TESS data available for a baseline of at
			least two years; the 27 brightest are shown here due to space
			constraints.  Each panel shows one sector of TESS data, and is
			phased to its deepest minimum in flux.  Each panel's title shows
			the TIC identifier and approximate period in hours.  Text insets
			show the TESS sector numbers, which generally span two years, or
			at least 1{,}000 cycles.  The vertical scale is fixed across
			sectors to clarify shape changes.  Gray circles are raw 2-minute
			data; colored circles bin to 300 points per cycle. 
		}
		\label{fig:evoln}
	\end{center}
\end{figure*}

\begin{figure*}[!tp]
	\begin{center}
		\centering
		\includegraphics[width=0.98\textwidth]{f5.pdf}
		\vspace{-0.3cm}
		\caption{
      {\bf LP 12-502 (TIC~402980664) light curve}, where each time
      chunk represents one TESS orbit.  Data were acquired in Sectors
      18-19, 25-26, 53, and 57-58.  Flares are drawn in gray.  The
      light curve is binned to 15-minute intervals so that there are
      96 points per day, and each point is connected by a line.  Data
      gaps longer are not interpolated; if data are missing, nothing
      is plotted.  The red vertical lines highlight apparently
      instantaneous state changes in the shape of the dip pattern.  
		}
		\label{fig:lplc}
	\end{center}
\end{figure*}


\subsection{Evolution over two year baseline}

Figure~\ref{fig:evoln} shows ``before'' and ``after'' views of 27 CQVs
for which TESS 120-second cadence observations were available at least
two years apart.  Such a baseline was available for 32 of the
confirmed \ngoods\ CQVs in our catalog; for plotting purposes we show
the brightest 27.  We have phased each sector to its own local minimum
because for most of the sources we do not know the period at the
precision necessary to be able to accurately propagate an ephemeris
over two years.  The achievable period precision, $\sigma_P$, can be
estimated as
\begin{equation}
  \sigma_{P} = \frac { \sigma_{\phi} P } { N_{\rm baseline} },
  \label{eq:periodprecision}
\end{equation}
for $N_{\rm baseline}$ the number of cycles in the observed baseline
and $\sigma_{\phi}$ the phase precision with which any one feature
(e.g.~a dip, or the overall shape of the sinusoidal envelope) can be
tracked.  Assuming $\sigma_\phi$$\approx$$0.02$ and a 20-day baseline
over a single TESS sector yields
$\sigma_{P}$$\approx$$0.25^{+0.38}_{-0.14}$\,minutes for the
population shown in Figure~\ref{fig:evoln}; propagated forward 1{,}000
cycles yields a typical ephemeris uncertainty range of 2-11\,hours.
Measuring the periods for each sector independently, there did not
seem to be any significant ($>$3$\sigma$) period changes.  This
implies an absolute period stability of $\lesssim$0.1\% over the
two-year baseline.

A few objects in Figure~\ref{fig:evoln}
show clear signs of
the CQV phenomenon in one sector, and marginal or non-existent signs
in the other.  While there is subjectivity in this assessment, to our
eyes cases for which at least one sector would be flagged as
``ambiguous'' include
TIC~368129164 (Sector 23 might be labeled an EB),
TIC~177309964 (Sector 38 would be simply a rotating star),
TIC~404144841 (Sector 38 looks like a rotating star),
TIC~201898222 (Sector 3 looks like a rotating star),
TIC~144486786 (Sector 32 might be an RS CVn),
and
TIC~38820496 (Sector 28 might be an RS CVn).
TIC~193831684, assessed on a single-sector basis, would probably be
labeled an eclipsing binary---in fact, \citet{2021ApJ...912..123J}
already gave this source such a label.  However,
based on the the shape evolution between Sectors 13 and 39, it is a
CQV.  Based on the fraction of sources overall that ``turned off'',
the observed shape evolution implies that CQVs have a duty cycle of
$\approx$75\%.  This type of correction is worth including in
population-level estimates of how intrinsically common CQVs are in the
low-mass stellar population \citep[e.g.][]{2022AJ....163..144G}.



\begin{figure*}[!t]
	\begin{center}
		\centering
		\includegraphics[width=0.98\textwidth]{f6.pdf}
		\vspace{-0.3cm}
		\caption{
			{\bf Evolution of LP 12-502} ($P$=18.6\,h) at fixed period and
			epoch over three years.  Each panel shows one (stacked) TESS
			orbit; small text denotes relative cycle number.  There are 200
			binned black points per cycle.  The TESS pointing law dictates
			the large time gaps between Cycles 64-248, 315-1233, and
			1264-1410; larger gaps tend to yield larger shape changes.  The
			dips usually evolve over tens to hundreds of cycles.  However
			cycles 1233-1264 show a dip that ``switched'' from a depth and
			duration of 3\% and 3\,hr to 0.3\% and 1\,hr over less than one
			cycle (cf.~Figure~\ref{fig:lplc}).
		}
		\label{fig:lp}
	\end{center}
\end{figure*}

%\begin{figure*}[!t]
%	\begin{center}
%		\includegraphics[width=0.45\textwidth]{ANNOTATED_resid_TIC_402980664_P18.5611_2min_phase_timegroups.png}
%	\end{center}
%	\vspace{-0.4cm}
%	\caption{
%		{\bf Alternative view of the evolution of LP 12-502}
%		(Figure~\ref{fig:lp}), arranged to emphasize changes in transit
%		times.  There are 200 binned black points per $P$=18.5611\,hr cycle; a two-harmonic
%		sinusoid has been subtracted from the PDC\_SAP light curve.
%		Vertical gray lines are underplotted to help guide the eye to instances
%		in which preferred dip phases synchronize over long baselines.
%		The orange and green lines guide the eye to where dips
%		change the positions of their local minima; orange
%		lines have periods slightly less than $P$, the green line has
%		a period greater than $P$.
%	}
%	\label{fig:lp2}
%\end{figure*}


\begin{figure*}[!t]
	\begin{center}
		\subfloat{
			\includegraphics[width=0.46\textwidth]{f7a.pdf}
			\includegraphics[width=0.46\textwidth]{f7b.pdf}
		}
		\vspace{-0.2cm}
		
		\subfloat{
			\includegraphics[width=0.46\textwidth]{f7d.pdf}
			\includegraphics[width=0.46\textwidth]{f7c.pdf}
		}
	\end{center}
	\vspace{-0.4cm}
	\caption{
		{\bf River plots of the LP 12-502 light curve}, showing (clockwise
		from top-left) Sectors 18-19, 25-26, 53, and 58-59.  A
		two-harmonic sinusoid has been subtracted to highlight the sharp
		dips.  In Sectors 25-26 (cycles 248-315), periods are visible at
		the fundamental period of 18.5611\,hr, as well as at faster and
		slower relative periods (e.g. $\phi$$\approx$0.3 and
		$\phi$$\approx$0.1).  Similar instances of multiple simultaneous
		periods are also visible in Sector 53 and 58-59.  White chunks
		denote missing data.
	}
	\label{fig:lpriver0}
\end{figure*}


\subsection{Evolution over adjacent sectors, \& LP 12-502}

A small fraction of our objects were in regions of sky with many
sectors of adjacent TESS data.  Out of this already small sample, LP
12-502 (TIC 402980664; $d$=21\,pc, $J$=9.4, $T$=11.1) stood out due to
the quality and content of its data.  We discuss another interesting
source, TIC~300651846, in Appendix~\ref{app:tic3006}.  In this
section, we first present the LP 12-502 observations and then
highlight their implications.


\subsubsection{LP 12-502 observations}
\label{subsec:lpobservations}

Figure~\ref{fig:lplc} shows all available data for LP 12-502 from TESS
Sectors 18, 19, 25, 26, 53, 58, and 59, split into successive orbits.
The star was observed at 120-second cadence whenever it was observable
by TESS.  We binned the light curve to 15-minute intervals for visual
clarity, and required at least one (120-second cadence) flux
measurement per bin.  Points more than 2.5$\sigma$ above the median
are drawn in gray, also for visual clarity.  Missing data are not
drawn.  Figure~\ref{fig:lp} then shows the same data, but stacked into
successive TESS orbits spanning half a lunar month each.  Finally,
Figure~\ref{fig:lpriver0} shows ``river plots'' of the same data,
split into similar intervals: the Sector 18-19 data, 25-26 data, 53
data, and 58-59 data.  The river plots are subject to one additional
processing step: we fitted and subtracted a maximum-likelihood
two-harmonic sinusoid independently from the Sector 18-19 data, 25-26
data, and 53, 58, and 59 data in order to accentuate changes in the
dip timing and structure.

The average period, determined by measuring the PDM peak period over
each sector independently, was $\langle P \rangle = 18.5560$\,hr.  The
range between the maximum and minimum sector-specific periods was
measured to be about one minute.   However, in detail, a period shift
of $\pm$1\,minute would yield major phase drifts over the baseline;
that time interval corresponds to $\approx$1/1000$^{\rm th}$ of a
period, and we have observed 1500 cycles.  By fine-tuning over a grid
in period, we found $P=18.5611 \pm 0.0001$\,hr, which seems to track
certain features in the LP 12-502 light curve well over the entire
dataset.

What exactly is observed?  For the first 64 cycles, the star shows a
pattern reminiscent of an eclipsing binary, with four obvious local
minima.  We dub these dips $\{ 1, 2, 3, 4 \}$ at phases $\{ -0.28,
-0.08, 0, 0.25 \}$, respectively.  Over cycles 0-64, the depth of dips
1 and 3 remain roughly fixed.  However dip 4 decreases in depth by
about 2\%, while dip 2 increases in depth, by about the same amount
(see Figure~\ref{fig:lp}).  During cycles 48-64, a fifth dip may also
be emerging, in the main ``large'' dip group.

There is then a 6-month (184 cycle) gap to Cycles 248-315, which show
two highly structured dip complexes, plus a small leading dip.  The
single leading dip is present at the same phase as in cycles 0-64, and
is therefore likely to be the same struture.  Along a similar line of
logic, it seems plausible that the first ``dip complex'' during cycles
248-264 represents an evolution of the initial complex seen during
cycles 0-64, though with reduced depth.  During cycles 266-310, an
additional local minimum develops between the two complexes; this
feature is best visualized on the river plots
(Figure~\ref{fig:lpriver0}), and we describe its (shorter) period
below.

The second dip complex during cycles 248-315 shows the most
substructure.  During e.g.~cycles 283-298, this single complex shows
six local minima.  The deepest dip is sharp: it shows a flux excursion
of 3.5\% over about 22 minutes (0.02\,$P$), which is the steepest
slope exhibited anywhere in the LP~12-502 dataset.  After the sharp
dip, there is a roughly exponential fall-off spanning about a quarter
of a period, punctuated by coherent local minima and maxima which in
detail (Figure~\ref{fig:lpriver0}) have slightly longer periods than
the sharp dip.  The sharp leading dip only decreases in depth
following a sudden state-switch at BTJD 2030.7 (cycles 299-315), which
happens during a flare (Figure~\ref{fig:lplc}).  The trailing dips
remain thereafter.

Sectors 53--58 (cycles 1233-1481) are comparatively tame; they showed
only four to six dips per cycle.  Some dips remain stable in depth and
duration over this five month interval.  Other dips grow, like the one
at $\phi = +0.06$ between cycles 1499 and 1481.  Other dips, such as
the one at $\phi = +0.12$ in cycles 1233-1264, disappear entirely.
The most dramatic state switch occurs during cycles 1233-1247, when a
large dip ``switches'' from a depth of 3\% and duration of 3 hours to
a depth of 0.3\%, and a duration of 1 hour.


\subsubsection{Lessons from LP 12-502}
\label{subsec:lplessons}

{\sc State-switches reveal dip independence}---The state-switches seen
in cycles 1233-1247 and 299-315 confirm that dips can disappear in
less than one cycle, a point which has been previously appreciated
\citep{2017AJ....153..152S}.  What is new in these particular changes
is that the morphology changes show that the dips can be {\it
independent} and {\it additive}.  For example, throughout cycles
1233-1264, there are three local minima between phases of 0 and 0.3.
They all have identical ingress times.  The shape change during the
transition implies that the leading dip that ``turned off'' (reduced
its depth and duration), while the trailing two dips remained fixed.
This is visible in Figure~\ref{fig:lp}.  In other words, the
structures producing these dips are independent to the degree that one
can undergo a severe change while the others remain identical.  The
state switches during cycles 248-264 and 299-315 share the same
characteristic: it is always the {\it leading} dip of a complex that
``switches off'', leaving the (fixed-depth) trailing dips in its wake.

{\sc Slow growth; rapid death}---Although there are a few instances in
which we observe dips switch off over less than one cycle, dip growth
seems to happen more slowly.  For instance, the dip that grows between
phase 0 and 0.1 between cycles 260-290 begins to become visible around
BTJD 1993.2, and growths in depth by about 2\% over about six cycles,
to become visible by eye by BTJD 1997.7.  The evolution of this
particular dip is most clear in the river plots.  The evolution of the
latter dip group in cycles 1410-1481 is another example of this slow
mode of dip growth.

% 16 px vs 310 px for a full phase
% Based on our SED fitting $R_\star=0.369\,R_\odot$ star, and from the
% isochrone mass, $M_\star=0.215\,M_\odot$ 

{\sc Dip durations}---The shortest dip duration for any of the
individual LP 12-502 dips seems to be $\approx$0.06\,$P$ $\approx$
1.08\,hr.  In comparison, using the stellar radius and mass derived in
Section~\ref{subsec:starprops}, the characteristic timescale $T_{\rm
dur} \equiv R_\star P / (\pi a)$ for the transit of a point-source at
corotation is 1.02$\pm$0.10\,hr.  This means that while some of the LP
12-502 dips are sufficiently long to require structures that are
extended in orbital azimuth, the durations of other dips are
consistent with effective radii for the occulting material $R_{\rm
eff} \ll R_\star$.  This implies $a/R_\star \approx 5.8$ for this
material; the analogous timescale at the stellar surface is about six
times slower.

{\sc Dip periods}---Most of the LP 12-502 dips repeat with a period of
$P=18.5611 \pm 0.0001$\,hr.  However the river plots
(Figure~\ref{fig:lpriver0}) reveal that the light curve has specific
dips with simultaneous but distinct periods.  For instance, in sectors
25-26, the local minimum that develops around cycle 262 has a period
faster than the mean period by $\approx$0.1\%, while some of the
trailing local minima in the main dip complex have periods slower than
the mean period, by $\approx$0.04\%.  In addition to the fundamental
period, we were able to identify at least four distinct periods shown
by specific dips over the full Sectors 18-59 dataset, including
periods at 18.5683, 18.5672, 18.5473, and 18.5145\,hr, with a typical
measurement uncertainty of $\approx$0.0002\,hr.  If each period
corresponds to an individual dust or gas clump, then this implies that
multiple distinct clumps can orbit the star simultaneously, at
marginally different separations.



\section{Discussion}
\label{sec:discussion}


\subsection{Typical and extreme CQVs}
\label{subsec:extreme}

Figure~\ref{fig:catalogscatter} shows derived stellar properties for
our CQV catalog, and compares the CQVs against both the target star
sample (top panels), and against stars in the Pleiades
\citep{2016AJ....152..114R}.  Typical CQV masses span
0.1-0.4\,$M_\odot$, and typical ages span 2-150\,Myr.  This mass and
age range includes both fully convective stars
($M_\star$$\lesssim$0.25\,$M_\odot$), and stars with a combination of
radiative cores and convective envelopes
($M_\star$$\gtrsim$0.25\,$M_\odot$; \citealt{2018A&A...619A.177B}).
We found no obvious light curve morphology differences between the two
classes.  In terms of their rotation rates relative to the Pleiades,
the CQVs are among the more rapidly rotating half of M dwarfs. 

The closest CQV in our catalog is DG~CVn (TIC~368129164;
$d$$=$18\,pc), a member of AB Dor.  To our knowledge, this work is the
first time that it has been noted as a CQV.  The three brightest CQVs
are DG~CVn ($T$=9.3), TIC~405754448 ($T$=9.6), and TIC~167664935
($T$=10.3).  The shortest period belongs to TIC~201789285, at
3.64\,hr.  The longest period belongs to TIC~405910546, at 37.9\,hr.
If the latter source turns out to be an eclipsing binary, the
next-longest would be TIC~193831684 (31.0\,hr).

The lowest mass ($\approx$$0.12$\,$M_\odot$) belongs to TIC~267953787.
The catalog contains a few other stars with similar mass.  Given the
small number of sub-stellar mass objects in our target sample, future
studies of brown dwarf photometric variability might also yield CQVs,
though there could be degeneracies in interpretation with planetary
surface features such as clouds and latitudinal bands
\citep[e.g.][]{2021ApJ...906...64A,2022ApJ...924...68V}.


\subsection{CQVs and binarity}
\label{subsec:discbinary}

\subsubsection{Binary statistics}

In Section~\ref{subsec:binarity}, we found that a significant fraction
of the CQVs show indications of unresolved binarity.  Excess noise
above the Gaia single-source astrometric model is common
(\ngoodhighruwe/\ngoods\ high-quality CQVs with RUWE$_{\rm DR3}$$>$2),
as is the presence of multiple periods in the TESS light curves
(\ngoodmultperiodflag/\ngoods).  Elevated astrometric noise is almost
always accompanied by multiple detectable TESS periods
(\ngoodruweandmultperiod/\ngoodhighruwe high-quality CQVs).  The
latter point strongly corroborates the idea that most sources with
RUWE$_{\rm DR3}$$>$2 are intermediate separation binaries with
projected apparent separations below 1$''$, and projected physical
separations $\lesssim$50\,AU.

The multiplicity rate of M dwarfs near the Sun is 26.8$\pm$1.4\%
\citep{2019AJ....157..216W}.  Based on the same study, the peak of the
separation distribution decreases from $\approx$49\,AU for
$0.30$-$0.60$\,$M_\odot$ stars, to $\approx$11\,AU for
$0.15$-$0.30$\,$M_\odot$ M dwarfs.  The multiplicity fraction in our
CQV sample seems either consistent or perhaps marginally higher than
the field fraction.  This could plausibly be understood in a physical
framework in which the presence of intermediate-separation binaries
causes early disk dispersal, freeing the star to contract.  The CQV
phenomenon seems to require rapid rotation; thus if binary systems
tend to produce more rapidly rotating stars, we would expect a sample
of CQVs to have a larger binary fraction than the field.


\subsubsection{Do K dwarf CQVs exist?}
\label{subsec:massive}

To date, the only stars reported to show the CQV phenomenon are M
dwarfs, with typical stellar masses $\lesssim$0.3\,$M_\odot$
\citep{2017AJ....153..152S,2022AJ....163..144G}.  However the two most
massive CQVs in our sample, TIC~405754448 and TIC~405910546, appear to
have masses of $\approx$0.82\,$M_\odot$ and $\approx$0.60\,$M_\odot$
respectively.  The next-highest masses in our sample are
$\approx$0.40\,$M_\odot$.

Based on their locations locations in the color--absolute magnitude
diagram and membership in LCC, both of the K dwarfs do indeed appear
to be high mass.  However in detail, both are subject to ambiguities
in interpretation.  The TIC~405910546 light curve has a unique shape,
suggestive of an eclipsing binary.  Independently, TIC~405910546 was
one of only \nrvscatterflag\ CQVs flagged with a Gaia DR3 radial
velocity scatter exceeding 20\,\kms.  Combined, these factors suggest
that TIC~405910546 might be a pre-main-sequence eclipsing binary; it
should be studied further to clarify this classification.  For
TIC~405754448, we believe that the source {\it is} an unresolved
binary, because RUWE$_{\rm DR3}$=6.8, and because the raw light curves
in Sectors 11, 37, and 38 show an additional $\approx$6.5\,day,
$\approx$0.3\% amplitude sinusoidal signal suggestive of an unresolved
photometric companion.  If TIC~405754448 is a K+M binary, then the
flux ratio between the primary and secondary would be expected to be
$\approx$10:1.  It would be challenging for the M dwarf to produce the
observed 12.9\,hr signal with such a flux deficit, but not impossible.
While both of these objects suggest that the CQV phenomenon may extend
up in mass to pre-main-sequence K dwarfs, more data are needed to
fully substantiate this claim.

\subsubsection{An astrophysical CQV false positive: TIC~435903839}

We originally classified TIC~435903839, with RUWE$_{\rm DR3}$=17.7, as
an ``ambiguous'' CQV with a 10.8\,hr period, because this period
minimized the dispersion in the phase-folded light curve.  More
careful inspection however showed that this source is an impostor:
this source is a photometric blend of two ordinary rotating stars with
$P_0$=3.60\,hr, and $P_1$=5.41\,hr, with a beat period $(P_0^{-1} -
P_1^{-1})^{-1}$ of 10.8\,hr.  This is a novel false positive scenario
for CQVs: two rapidly-rotating stars with near-integer ratios of
rotation periods.  The beat between the two rotation signals produces
the apparent CQV signal.  Such false positives can be excluded through
careful accounting of all peaks in a periodogram.  For instance,
TIC~435903839 shows a peak at 16.27\,hr, which is not an integer
multiple of the dispersion-minimizing 10.82\,hr period.

\subsubsection{Multiple CQVs in the same system: TIC~425937691 and TIC~142173958}

TIC~142173958 and TIC~425937691 both show evidence for two separate
CQV signals in their TESS light curves.  For TIC~142173958, the
signals are at 11.76\,hr and 12.84\,hr.  For TIC~425937691, they are
at $\approx$4.82\,hr and $\approx$3.22\,hr, near the 3:2
commensurability.  Given that both sources have two photometric
signals and elevated RUWEs, they are probably unresolved binaries
consisting of two independent CQVs.  Recent work has shown that the
orbits of binaries closer than $\lesssim$700 AU tend to be aligned
with their planetary systems \citep[e.g.][]{2022AJ....163..207C}.  If
we assume that observing CQV variability similarly requires near
edge-on viewing geometries, then we might expect a sufficiently close
binary with one CQV to also have another, since the spin axes in
sufficiently close binaries would tend to be aligned.  In this
scenario, the two periods of TIC~425937691 being within $\lesssim$1\%
of the 3:2 period commensurability would simply be a coincidence.


\subsection{CQVs are quasiperiodic}

A periodic signal repeats exactly; the CQVs do not
(Figure~\ref{fig:evoln}).  While their periods appear to remain
constant to within measurement precision over thousands of cycles, the
light curve shapes evolve over 10 to 1{,}000 cycles.  They are
therefore {\it quasi}periodic.  While this observation is consistent
with the analyses by \citet{2022AJ....163..144G} and
\citet{2023ApJ...945..114P}, it marks a qualitative departure from the
``persistent'' {\it vs.} ``transient'' flux dip distinction described
by \citet{2017AJ....153..152S}, since all CQV dips seem to be
transient over timescales of more than 1{,}000 cycles
(Figure~\ref{fig:evoln}).

With that said, one might expect a truly quasiperiodic process to be
able to explore all phase angles with equal weight.  LP 12-502, and
perhaps other CQVs, seem to have preferred phases.  For LP 12-502, all
of the dips happen over phases corresponding to only two thirds of the
period (Figure~\ref{fig:lp}).  The remaining third seems to be ``out
of limits'' for any dipping material.  This could be evidence that the
stellar magnetic field is strongly asymmetric, and can hold extrinsic
material at corotation, but only over two thirds of the equatorial
circle.  Alternatively, the source of the material (e.g. a
planetesimal swarm) might be distributed over an arc of the same
angular extent (240$^\circ$).  We favor the former explanation, for
reasons discussed below.



\subsection{Dip asymmetries and dust geometries}

The asymmetry of a dip can help diagnose the optical depth of the
occulting material as a function of orbital phase angle.  Sharp
leading edges with trailing exponential egresses for instance have
been previously seen for transiting exocomets and disintegrating rocky
bodies
\citep[e.g.][]{2012ApJ...752....1R,2012A&A...545L...5B,2015Natur.526..546V,2019A&A...625L..13Z}.

Examining Figure~\ref{fig:cqvs}, it is not obvious whether CQVs as a
whole show any preference for sharper ingresses, or sharper egresses.
In some cases (e.g. TIC~425933644), the continuum itself is not
well-defined, and so the meaning of ``ingress'' and ``egress'' are not
clear.  In others, such as Sector~36 of TIC~89463560, there is a
single clear sharp ingress with an exponential egress, which could be
fitted using e.g.~a model analogous to those used for exocomets
\citep[e.g.][]{2019A&A...625L..13Z}.  The main quantities of interest
in such models would be the exponential decay time- and therefore
length-scale, as well as the impact parameter and the inferred transit
depth (and its implications for the equivalent ``radius'' of the
transiting cloud).  We briefly explored such models, until it
became clear that careful modeling of sources such as LP
12-502 merits its own in-depth study.  Connections could likely also
be made to the toroidal geometries that can be produced by outflowing
atmospheres of transiting planets
\citep[e.g.][]{2019ApJ...873...89M,2022ApJ...926..226M}.



\subsection{Dust, or gas?}

We believe that the most likely scenarios to explain CQVs are either
the dust clump scenario or the prominence scenario
(Figure~\ref{fig:f1}).  Both invoke clumpy material that would be
centrifugally supported at the corotation radius.  The prominence idea
has a longer history, based on analogy with quiescent
prominences/filaments observed to exist in the solar corona for up to
a few weeks \citep[see][]{2015ASSL..415.....V}.  In an extrasolar
context, spectroscopic detections of transient Balmer- and
resonance-line absorption seen for stars such as AB~Dor and Speedy~Mic
\citep[e.g.][]{1989MNRAS.238..657C,1993MNRAS.262..369J,2006MNRAS.365..530D,2016MNRAS.463..965L}
led to the interpretation that the data could best be explained by
similar structures: cold, minimally ionized hydrogen clouds or
filaments that scatter chromospheric emission from the star to produce
the observed spectroscopic line variations
\citep[see][]{1989MNRAS.238..657C}.  The short-term mechanical
stability of such gas configurations is theoretically plausible
\citep{2000MNRAS.316..647F,2022MNRAS.514.5465W}, and the
interpretation of this class of observations seems at least somewhat
secure.

A clear link between the dense gas clumps (prominences) that likely
exist around rapidly rotating low-mass stars and the CQV phenomenon
has yet to be made.  A simple visual examination of the TESS light
curves for five prominence-hosting systems studied by
\citet{2019MNRAS.482.2853J}--AB~Dor, Speedy~Mic, LQ Lup, HK Aqr, and
V374 Peg--revealed no obvious CQV behavior, though all show
differential evolution, and Speedy~Mic shows two closely-spaced
periods and a strong beat.  Spectroscopically observable prominences
do not imply photometric CQV-like dips.

The key difference between the prominence and dust clump scenarios is
in whether the occulting material of interest is neutral hydrogen, or
dust.  In the phrasing of the ``frozen flux'' condition of ideal rigid
field magnetohydrodynamics, the tendency of both to become trapped at
the corotation radius in the equatorial plane is tied to how of the
four relevant forces (gravity, Lorentz, inertial Coriolis, and
inertial centrifugal), the Lorentz and Coriolis only act perpendicular
to field lines, while gravity and the centrifugal force are in balance
at $R_{\rm cr}$ \citep[see][Sec.~2]{2005MNRAS.357..251T}.  The
magnetic field strength is only relevant in this formulation of the
system in that we must have $ R_{\rm cr} < R_{\rm Alfven}$ in order
for closed loops to exist that can support prominences
\citep[e.g.][]{1985Ap&SS.116..285N,2019MNRAS.482.2853J}.  In detail
however, the locations of such magnetic fixed points depends on the
star's magnetic field topology.


\subsubsection{Gas absorption microphysics}

CQVs show broadband flux variations that can be 1-2$\times$ deeper in
the blue than in the red
\citep{2017PASJ...69L...2O,2020AJ....160...86B,2022AJ....163..144G,2023MNRAS.518.2921K}.
For neutral hydrogen to produce this effect in absorption, its opacity
must be larger in the blue than in the red.
\citet{1992oasp.book.....G} suggests that this might be
microphysically possible: while bound-bound absorption provides
opacity only at narrow resonant lines, the hydrogen opacity due to
bound-free absorption is ``jagged'' \citep[see][Figure 8.5 and
Eq.~8.8]{1992oasp.book.....G}, such that at temperatures of
$\approx$3{,}000\,K to $\approx$10{,}000\,K the observed
chromaticities might be reproduced.  While bound-free absorption of
${\rm H}^{-}$ seems like it should be important at such cool
temperatures, in detail this opacity source has the {\it opposite
sign} from what is required to produce deeper dips in the blue than in
the red.  Thomson scattering is similarly ruled out as a relevant
opacity source, because it is gray.  The final plausible alternative
opacity source is dust, which has the requisite larger absorption
cross-section in the blue than the red \citep{1989ApJ...345..245C}.

An instructive point of comparison is the rapidly rotating magnetic B
star, $\sigma$~Ori~E, which shows dips that are deeper in the blue
than in the red \citep{1977ApJ...216L..31H}.  Photometric and
spectroscopic observations of this star have been understood in terms
of a warped torus of circumstellar material, analogous to the
geometries we are discussing for much cooler M dwarfs
\citep{1978ApJ...224L...5L,1985Ap&SS.116..285N,2005ApJ...630L..81T}.
This material is unlikely to be dust, due to the relevant sublimation
timescales.  The opacity is instead thought to be sourced from
bound-free absorption from neutral hydrogen
\citep{1985Ap&SS.116..285N}, although to our best knowledge direct
evidence for this conclusion has yet to be acquired.  Separate and
smaller-amplitude emission in $\sigma$~Ori~E may also come from
electrons scattering photospheric light toward the observer when the
clouds are not in transit \citep{2022MNRAS.511.4815B}.

Given these complexities, it seems important for a future theoretical
study to be conducted to determine to what degree the observed
chromaticities in CQVs match, or do not match, expectations from
radiative transfer.  This issue has a key ability to resolve the
question of whether the CQVs are explained by dust or by gas, which
has bearing on whether the material producing the dips is coming from
the star, or whether it is a byproduct of the protoplanetary disk.


%\subsubsection{Why corotation?}

% The stable point only exists if R_A > R_K.  For radii between R_A
% and R_K in such cases, consider a magnetic field line extending out
% beyond R_K.  Whenever R>R_K, the material is centrifugally supported
% (pushed out by the ~Ω^2R^2 centrifugal term in the corotating
% frame).  If R<R_K in such cases along a given field line, the
% material will fall back to the star.  So there are two conditions
% that must be met: the star must spin fast enough, and closed
% magnetic loops must extend beyond the corotation radius.  (This is
% all spelled out by Nakajima1985, and Petit2013).


\subsubsection{The lifetime constraint}

Independent of the microphysical basis for gas absorption, the
observed lifetime of the CQV phenomenon could provide another
dimension to discern between the gas {\it vs.} dust clump scenarios.
Based on the available statistics from e.g.
\citet{2022AJ....164...80R} and references therein, it seems plausible
that CQV occurrence decreases in time, bottoming to zero well before
the $\approx$700\,Myr age of Praesepe.  This is odd in the context of
the prominence scenario, because pre-main-sequence M dwarfs spin up
over the first $\approx$$10^8$\,yr; prominences might even be expected
to be {\it more} common at the age of the Pleiades than for younger
stars.  This broadly assumes that any star that can support
prominences will do so, and that the magnetic topology of rapidly
rotating M dwarfs at 100\,Myr is similar to that of rapidly rotating M
dwarfs at say, 10\,Myr.  Based on the degree to which M dwarfs remain
rapid rotators over the first few gigayears of their lives
\citep[e.g.][]{2022AJ....164...80R,2022ApJ...936..109P}, it seems
surprising in the gas clump (prominence) scenario that very few stars
older than 150\,Myr have been seen to exhibit the phenomenon.  In the
dust clump scenario this is not a problem, because if the dust were
externally sourced, it would be expected to have a finite supply.


\subsection{Planets or planetesimal swarms near corotation?}

Close-in planets are common around M-dwarfs; studies from Kepler have
shown that high-mass M  dwarfs have $\approx$0.1 planets per star with
sizes between 1-4\,$R_\oplus$ and orbital periods within 3 days
\citep{2015ApJ...807...45D}.  The number increases to $\approx$0.7
planets per star considering planets out to $P$$<$10\,days.
Extrapolating to all small close-in planets with say 0.1-4\,$R_\oplus$
and within 10 days, it is reasonable to expect on average one close-in
planet per M dwarf.

In the context of planet formation, the stopping location for the
inner-most planet is set by the location of the protoplanetary disk's
truncation radius \citep{2018haex.bookE.142I}.  Although this disk
truncation radius is defined by equating the magnetic pressure from
the stellar magnetosphere with the ram pressure form the inflowing
gas, it also tends to coincide with the Keplerian corotation radius
for low accretion rates
\citep{2016ARA&A..54..135H,2022MNRAS.510.5246L}.  Within models that
have compact multiplanet resonant chains migrating inward due to gas
drag, the inner-most planets therefore arrive at $\approx$5-10 stellar
radii before the disk becomes depleted.

Given this context, it is tempting to try to interpret features of the
CQV light curves in terms of the possible presence of either close-in
exoplanets or close-in planetesimals.  Rocky planets this young could
have molten global magma oceans analogous to those that existed on the
Earth and Moon \citep[see][]{2022arXiv220310023L}, and thus might be
undergoing significant outgassing and atmospheric escape.  The system
would then be analogous to the Jupiter-Io plasma torus
\citep[e.g.][]{1980JGR....85.1171N}, although with magnetic fields
stronger by a factor of $\approx$1{,}000.  However we emphasize that
while this type of configuration seems {\it a priori} plausible, no
direct evidence currently supports it.

In systems like LP 12-502, the observed variability would perhaps
drive one toward a picture of a disintegrating planetesimal swarm,
rather than just a single planetesimal.  The need for more than just
one launching body is because we see dips appear and disappear at many
different phases, but with roughly the same period.  The observed
changing dip depths in response to flares would be understood through
the planetesimal outflow rates being stalled by changes in the local
magnetic field or coronal plasma density.  The sizes of these
purported planetesimals would need to be $\lesssim 1$\,$R_\oplus$,
based on the non-detections of their transits, analogous to e.g.
K2-22 \citep{2015ApJ...812..112S} or KOI-2700
\citep{2014ApJ...784...40R}.  Typical sizes of bodies in the asteroid
belt for instance span hundreds of meters to a few kilometers.

There are at least three problems with this disintegrating
planetesimal swarm concept.  First, the model would predict that
certain orbital phases would produce recurrent dips if observed over
sufficiently long baselines, because the launching planetesimal would
be massive enough to remain in orbit, while stochastically ejecting
material.  An analogous system would be K2-22
\citep{2015ApJ...812..112S}.  For most CQVs (Figure~\ref{fig:evoln}),
the data are in tension with this expectation.  A second issue is that
many of the dips show asymmetries in the wrong direction relative to
the naive expectation of a trailing comet tail.  Invoking
non-exponentially decaying dust distributions as a function of azimuth
might be one way out, but such an idea lacks firm theoretical footing.
Finally, dynamical stability is a concern: in a model in which any new
dip is ``explained'' by invoking a new planetesimal, one might
eventually pack the corotation radius to a degree where instability
over short timescales would be guaranteed.

It is certainly possible that exoplanets or exoplanetesimals could end
up being connected to the CQV phenomenon, for instance as a possible
sources for the occulting dust or gas.  However for the time being,
none of the CQVs that we have studied in this work show clear evidence
for the nearby planets that we statistically expect should be common.


\subsection{From dippers to debris disks}
\label{subsec:discdippers}

In identifying the two candidate CQVs with outlying SEDs
(TICs~193136669 and TIC~57830249; Section~\ref{subsec:irexcess}), we
were prompted to reconsider our light curve-based labeling, and
ultimately concluded that these sources are dippers.  This episode
suggests that there could be overlap between CQVs and dippers.  It is
also worth emphasizing that our labeling of e.g.~TIC~57830249 was
based on a single sector of TESS data (Sector~36) when its behavior
was relatively periodic and it showed dip depths of a few percent.
However, in other TESS sectors (e.g. Sector~10), this source looks
completely different, varying in apparent flux by a factor of two,
with no discernible periodicity at all.

Assessing these results against the backdrop of our increasing
understanding of dippers
\citep[e.g.][]{2014AJ....147...82C,2016ApJ...816...69A,2021ApJ...908...16R,2022ApJS..263...14C,2022MNRAS.514.1386G},
it is clear that the loss of an infrared excess is associated with
strong changes in a star's optical variability.  It is reasonable to
imagine connections between CQVs and dippers: both classes of object
can show transient flux dips that are relatively narrow in duration.
The dips in both are probably associated with clumps of dust or gas.
However the CQV dips are both more periodic and less deep than those
of dippers; they also display far less transience over timescales of a
few to tens of cycles.  This is presumably because CQVs have
demonstrably less dust than (most) dipper stars.  At a population
level, the CQV stars are also older.  A common mystery between the
CQVs and dippers, in our own estimation, is how exactly the {\it
narrowness} of the dust clumps is produced.  It is not unreasonable to
imagine a similar mechanism operating for both types of object, tied
perhaps to a shared magnetic topology, or perhaps to a preference for
dust to inspiral to the star in clumped structures.


\subsection{Are half-cycles important?}

The interval of half a cycle period could be significant in the
context of CQVs for two reasons.  The first is that for material on a
circular orbit viewed edge-on, a half-cycle corresponds to the
interval between transit and secondary eclipse.  The second is that
the half-cycle also corresponds to the interval over which half of the
star's surface is visible.  Of the CQVs in Figures~\ref{fig:cqvs}
and~\ref{fig:evoln}, a fraction that seems greater than random might
exhibit a preference for showing dips or peaks that correspond to the
half-cycle interval.

One set of CQVs shows ``CQV behavior'' (some form of dip complex), but
which only lasts for half of any given cycle.  TIC~206544316 (Sector
2) is a canonical example; TIC~405910546 (Sector 38), TIC~167664935
(Sector 38), TIC~146539195 (Sector 5), TIC~118449916 (Sector 44), and
TIC~312410638 (Sector 38) seem to show the same tendency.

An independent set of CQVs exhibits small dips roughly half a cycle
after large dips; it is tempting to label the small dips secondary
eclipses.  Such sources include TIC~402980664 (Sector 25; relative to
the sharpest, deepest minimum), TIC~89463560 (Sector 36),
TIC~224283342 (Sector 2), and TIC~442571495 (Sector 12).  For the
particular case of TIC~224283342, this observation, coupled with Gaia
DR3 radial velocities suggesting a radial velocity scatter exceeding
20\,\kms, suggests that the system could be a low-mass eclipsing
binary transiting a giant polar spot, analogous to TOI-3884
\citep{2022A&A...667L..11A}.

In general however, it is challenging to interpret the significance of
the potential ``half-period regularities'';  the CQVs exhibit such a
wide range of variability that for seemingly any regularity one might
posit, it is easy to come up with counter-example objects which do not
follow the trend.  In other words, these half-cycle CQVs could simply
be a product of the human tendency to pattern match.

\begin{figure}[!th]
	\begin{center}
		\centering
		\subfloat{
			\includegraphics[width=0.48\textwidth]{f9a.pdf}
		}
			
		\vspace{-0.35cm}
		\subfloat{
			\includegraphics[width=0.48\textwidth]{f9b.pdf}
		}

		\vspace{-0.35cm}
		\subfloat{
			\includegraphics[width=0.48\textwidth]{f9c.pdf}
		}

		\vspace{-0.1cm}
		\caption{
      {\bf The magnetic B star connection.}
      $\sigma$~Ori~E and HD~345439 (top row) are magnetic B stars with
      predominantly dipolar magnetic fields known to host
      circumstellar plasma tori.  HD~37776 and HD~64740 (middle row)
      are analogous magnetic B stars with field toplogies dominated by
      high order multipoles.  The bottom row compares the latter
      systems against the ``best-matching'' CQV light curves, selected
      by eye from Figure~\ref{fig:cqvs}.  CQVs have light curves that
      are visually similar to the topologically complex magnetic B
      stars.  Stellar masses rounded to one significant figure are
      given in the lower right of each panel; the star, TESS sector,
      and period are listed in each subtitle.
		}
		\label{fig:bstar}
	\end{center}
\end{figure}


\subsection{Mass flux estimate}
\label{subsec:massflux}

We can estimate the mass of a transiting cloud by first converting the
transit depth to an effective cloud radius, $R_{\rm cloud}$.  For most
CQVs in Figure~\ref{fig:cqvs}, this yields $\approx$2-20\,$R_\oplus$.
A minimum constraint on the number density follows by requiring the
cloud to be optically thick.  For cases like LP 12-502, this is
reasonable because the transit duration of the shortest dips implies
$R_{\rm cloud}\ll R_\star$.  Carrying out the relevant calculation
assuming the occulting material is dust grains 1\,$\mu$m in size,
\citet{2023MNRAS.518.4734S} reported minimum cloud masses of order
$10^{12}$\,kg (their Eq.~23), which scale linearly with both the
optical depth and dust grain radius.  This is comparable to a small
asteroid; the asteroid belt itself has a mass of order
$\approx$10$^{21}$\,kg \citep{2019Icar..319..812P}.  If the material
were cool gas prominences rather than dust, the mass would need to be
of order $10^{14}$\,kg \citep{1990MNRAS.247..415C}, about 100$\times$
larger than the requisite dust mass.

Our observations provide a direct measurement of how often dips appear
and disappear, both due to sudden state-switches, and due to more
gradual, secular evolution.  For instance, LP~12-502 showed three
``state-switch'' events over the six months of available TESS
observations, during cycles 248-264, 299-315, and 1233-1247.  In each
case, a dip ``turned off''.  It is plausible to imagine that these
events correspond to either mass being ejected from the corotating
clumps, or else perhaps being accreted onto the star.  In either case,
the corresponding $\dot{M} \equiv M\cdot {\rm d}N/{\rm d}t$
time-averaged over six months is $\approx$3$\times$$10^{-18}
M_\odot\,{\rm yr}^{-1} \approx 1\times10^{-12} M_\oplus\,{\rm
yr}^{-1}$.  Considered cumulatively over the $\approx$$10^8$ years for
which the CQV phenomenon is observed, this yields a cumulative moved
dust mass of $10^{-4}\,M_\oplus$, of order the Solar System's asteroid
belt.  If the occulting material is gas, the masses involved would be
of order 100 times larger.  For cases in which we observe the {\it
growth} of dips, such as the Sector~29 data for TIC~224283342, or
Sector~5 of TIC~294328885, the dip depths typically increase by of
order a few percent over ten to twenty days.  This growth rate yields
a mass flux one order of magnitude larger than the earlier estimate.


\subsection{Strengthening the magnetic B star connection}

\citet{2017AJ....153..152S} previously noted a possible connection
between the CQVs and rapidly rotating magnetic B stars such as
$\sigma$~Ori~E, which can have circumstellar gas clouds trapped in
corotation \citep{2005ApJ...630L..81T}.  The $\sigma$~Ori class is
distinct from Be-star decretion disks, which do not have any obvious
connection to the stellar magnetic field \citep{2013A&ARv..21...69R}.

An argument against the connection between CQVs and the $\sigma$~Ori~E
analogs is that the light curve of $\sigma$~Ori~E is much simpler than
those in Figure~\ref{fig:cqvs}, with only two broad local minima, and
one ``hump'' (Figure~\ref{fig:bstar}; see also
\citealt{2022ApJ...924L..10J}).  Within the model proposed by
\citeauthor{2005ApJ...630L..81T}, this is explained by the simplicity
of the star's magnetic field, which is well-approximated by a dipole,
similar to most other magnetic B stars
\citep{2007A&A...475.1053A,2009ARA&A..47..333D}.  The magnetic axis
needs to be tilted relative to the stellar spin axis in order to match
the qualitative behavior of both the broadband light curves, and the
line-profile variations seen in hydrogen, helium, and carbon
\citep{2012MNRAS.419..959O}.

Two exceptions to the rule of ``simple field topologies for magnetic B
stars'' are HD~37776 and HD~64740, which are B stars that were
previously known from spectropolarimetry to have field geometries
dominated by high order multipoles \citep{2011ApJ...726...24K}.
Reviewing the literature, we learned that recent TESS light curves of
these two B stars appear surprisingly similar to the CQV light curves
\citep{2020pase.conf...46M}.  The middle row of Figure~\ref{fig:bstar}
shows the phased TESS light curves for these two outlying magnetic B
stars, with by-eye ``best matching'' CQVs shown for comparison.  The
number of dips per cycle, the shapes of the dips, and the dip depths
relative to the sinusoidal envelope all appear to be similar.  The
implication of this surprising connection is that the
highly-structured M dwarf light curves may imply multipolar magnetic
fields -- since it is the complex fields which are the defining
characteristic of these two B stars, relative to other known magnetic
B stars like $\sigma$~Ori~E.  This is consistent with the
non-axisymmetric topologies that have been reported in a small number
of M dwarfs for which Zeeman Doppler Imaging is technically feasible
\citep[see the review by][]{2021A&ARv..29....1K}.

The physical similarity between the B stars and the M dwarfs
presumably has its origin in the existence of a ``centrifugal
magnetosphere'' \citep[see][]{2013MNRAS.429..398P}.  In other words,
both classes of objects probably satisfy the condition $R_{\rm A} >
R_{\rm cr}$, for $R_{\rm A}$ the Alfv\'en radius.  When this condition
holds, stable points exist wherever the outward centrifugal force is
balanced by the combined effect of gravity and magnetic tension along
closed field loops \citep{1985Ap&SS.116..285N}.  In the converse
``dynamical'' case, when  $R_{\rm A} < R_{\rm cr}$, any material
launched by the star returns to its surface over the free-fall
timescale.  A simple order of magnitude estimate assuming a dipole
field with $B_0\approx1\,{\rm kG}$ at the star's surface, a local
plasma number density $n\approx10^9\,{\rm cm^{-3}}$, and a plasma
temperature $10^6\,{\rm K}$ gives Alfv\'en radii of order a few times
the corotation radii, $R_{\rm cr}$.  This suggests that the existence
of a centrifugal magnetosphere is plausible in young, rapidly rotating
M dwarfs.

% NOTE to the arxiv crawlers:

% While there are some qualitative connections to Jupiter-Io, it's not
% perfect.  In that system, R_A>R_cr...  But the plasma torus exists
% near Io's 42 hour orbit ...and is nonetheless STILL FORCED into
% corotation at a 10hr period (presumably by Jupiter's B-field).  This
% type of structure is very different, because here we actually see
% the plasma at corotation!


\section{Conclusions}
\label{sec:conclusion}

In this work, we searched 120-second cadence TESS data collected from
2018 July to 2022 September for complex quasiperiodic variables
(CQVs).  The target stars were \nstarssearched\ K and M dwarfs within
150\,pc.  The selection function was $>$$80\%$ complete within 30\,pc,
and $<$$10\%$ complete at distances exceeding 100\,pc.

In our target sample, we found \ngoods\ objects that showed complex
quasiperiodic behavior over at least one TESS sector.  These \ngoods\
bona fide CQVs are listed in Table~\ref{tab:thetable}.  This table also
includes an additional \nmaybes\ ambiguous CQVs, whose designation is
less certain, and \ndebunked\ impostors.  We inferred ages for all but
two of the \nallcands\ objects based on association memberships; we
also derived temperatures and radii using SED fitting, and then
inferred stellar masses by interpolating against stellar evolutionary
models.  We caution that our selection function was not
volume-limited: the TESS 120-second stellar sample had a heterogeneous
selection function which may have been biased in favor of young stars
over field stars.  Previous work however has shown that $\approx$1-3\%
of M dwarfs younger than $\approx$100\,Myr show the CQV phenomenon
\citep{2016AJ....152..114R,2022AJ....163..144G}.

Analyzing the TESS light curves and stellar properties of our CQVs, we
draw the following conclusions.

\begin{enumerate}[leftmargin=*]
	%
    \item CQVs are quasiperiodic.  The mean periods remain fixed over
      the $>$1{,}000-cycle baseline of available observations; but the
      light curve shapes always evolve (Figure~\ref{fig:evoln}).
  %
    \item The same CQV can show multiple periods simultaneously.  LP
      12-502, for instance, showed dips with four distinct periods
      within $\pm 0.3\%$ of its fundamental period, sometimes
      simultaneously, and each lasting for up to 50 cycles
      (Figure~\ref{fig:lpriver0}).
  %
    \item CQVs evolve over timescales that are both secular
      ($>$100\,cycles) and impulsive ($<$1 cycle).  Dip growth seems
      to happen over durations of at least ten cycles, and slow dip
      decay can also occur.  ``State-switches'' correspond to dips to
      collapsing instantaneously, and are almost always linked with
      observed optical flares.  This suggests that the occulting
      material is sensitive to sudden changes in the magnetic field.
  %
    \item The rate of dip evolution can be used to place a
      model-dependent constraint on how much material is either being
      accreted or ejected during the state changes
      (Section~\ref{subsec:massflux}).  Order of magnitude estimates
      require at least an asteroid belt's worth of dust
      ($10^{-4}$\,$M_\oplus$) over $10^8$ years, or
      $\approx$$10^{-2}$\,$M_\oplus$ if the occulting material is gas.
  %
    \item The CQV phenomenon persists for $\gtrsim$150\,Myr, based on
      the existence of multiple CQVs in AB Dor, the Pleiades, and
      Psc-Eri (Section~\ref{subsec:starprops}).  It may even extend to
      200\,Myr, based on the one CQV we found in the Carina Near
      moving group (TIC~294328887; $\approx$200\,Myr).  However the
      lack of detected CQVs in the Hyades and Praesepe suggests that
      the lifetime of the phenomenon is limited to the first few
      hundred million years.
	%
    \item The duty cycle for CQVs seems to be $\approx$$75$\%, based
      on the fraction of bona fide CQVs that either turned on or turned
      off during TESS re-observations, two years after the initial
      observation.
  %
    \item Most CQVs are M dwarfs with masses
      0.1-0.5\,$M_\odot$.  Two sources, TIC~405754448 and
      TIC~405910546, have masses that appear to exceed the M dwarf
      limit.  However both are potentially binaries, and this may
      confuse our ability to accurate identify the source of the CQV
      signal (Section~\ref{subsec:massive}).  We encourage additional
      scrutiny of these objects in future work.
  %
    \item Surprising analogs to the CQVs exist in magnetic B stars,
      but only those with multipolar field topologies.  Since most
      magnetic B stars have dipolar magnetic fields, this suggests
      that the CQV dips and warps are similarly being sculpted by the
      stellar magnetic fields, and that the magnetic fields themselves
      are probably also multipolar.
  %
    \item The closest CQVs to the Sun are at distances of 15-20\,pc,
      and the brightest have $V$$\approx$12 ($J$$\approx$7.5).  We
      have found most of them in this work, since our CQV sample was
      $\gtrsim$80\% complete within 30\,pc.  The lack of CQVs in the
      volume-complete $<$15\,pc sample of 0.1-0.3\,$M_\odot$ stars
      analyzed by \citet{2021AJ....161...63W} is consistent with this
      estimate.  Expanding our analysis of the TESS data to the full
      frame images would yield a truly volume-limited selection
      function, and would expand the CQV census by about a factor of
      two within 50\,pc, and by a factor of ten within 100\,pc.
\end{enumerate}

While many questions remain, two in particular will be important for
clarifying what these objects might teach us in a broader
astrophysical context: {\it 1)} Is the eclipsing material responsible
for the phenomenon gas or dust?  {\it 2)} What sets the characteristic
clumping size for the circumstellar material?

The distinction between gas or dust is important because it will
clarify whether the CQV phenomenon is intrinsic, so that material
comes from the star, or extrinsic, so that it is sourced through some
generic evolutionary phase of debris disks.  This knowledge would in
turn propagate to our understanding of whether the phenomenon is
primarily teaching us about dust production and processing in gas-poor
disks, or whether it is teaching us about the ability of cold gas to
remain stable in hot stellar coronae for long durations.
Observationally, acquisition of medium- or high-resolution time-series
spectra holds a good chance at resolving the gas vs.~dust question.
Given our observed $\approx$75\% duty cycles, such data must be
acquired simultaneously with photometric time-series observations
(e.g. during TESS re-observation) in order for detections and
non-detections to be interpretable.

In both the gas and dust scenarios, CQVs are preferentially viewed
edge-on.  This implies that after correcting for the line-of-sight
inclination, roughly one third of low mass stars \citep[those that
rotate rapidly enough][]{2022AJ....163..144G} could trap circumstellar
material in the same way.  It also suggests that CQVs may
preferentially show transiting planets at larger distances than the
corotating material, though this may be dependent on whether the
magnetic and stellar spin axes tend to be aligned.  Given these
points, observational follow-up work should include searching for
outer transiting planets, and measuring equatorial velocities in order
to test whether the stellar inclination angles are indeed
preferentially edge-on.  Any source of empirical information on the
stellar magnetic field, whether from the Zeeman effect
\citep[e.g.][]{2021A&ARv..29....1K} or perhaps radio emission
\citep[e.g.][]{2015Natur.523..568H}, could also help clarify the
strength of the magnetospheres for these objects.

On the theoretical front, building a physical understanding what sets
the characteristic size scale of the clumping material would help
clarify why the light curves have the bizarre shapes that are
observed.  The relevant puzzles in plasma physics and radiative
transfer could perhaps be connected to our understanding of the
close-in rocky planets that are expected to be present around most of
these stars.  The challenges intrinsic to both the observational and
theoretical work seem worth the effort.



\acknowledgments
LGB gratefully acknowledges support from the Heising-Simons
51~Pegasi~b Fellowship.  A few thank-yous in particular (from LGB) are
owed to G.~Laughlin, A.~Mann, and J.~Spake for clarifications and
ideas, and to E.~Schaller and the 51 Peg science team for organizing
the annual summits.  In addition, this paper relied on data collected
by the TESS mission, which are publicly available from the Mikulski
Archive for Space Telescopes (MAST).  Funding for the TESS mission is
provided by NASA’s Science Mission Directorate.  TIC 402980664 in
particular was observed at 120-second cadence thanks to the TESS Guest
Investigator programs G022252 (PI: J.~Schlieder; Sectors 18, 19, 25,
26) and G04168 (PI: R.~Jayaraman; Sector 53).

%{\bf AUTHOR CONTRIBUTION STATEMENTS: PLEASE REVISE IF APPROPRIATE)}
LGB conceived the project and performed the dip-counting search, light
curve classification, cluster membership, SED, variability, and
secondary-period analyses.
RJ and SR performed the Fourier-based analysis and contributed
to light curve classification.
LR cross-examined the light curve classification, and contributed an independent SED analysis.
AD identified the magnetic B star connection.
LAH contributed to project design and to the interpretation of the light curves.
G\'AB acquired and maintained the servers used to run the dip-finding pipeline.
GRR is an architect of the TESS mission.
{\bf JNW?  MNG?}
All authors assisted in manuscript writing and revision.

%\clearpage

\software{
  %\texttt{arviz} \citep{arviz_2019},
  %\texttt{altaipony} \citep{ilin_flares_2021},
  \texttt{astrobase} \citep{2021zndo...1011188B},
  %\texttt{astroplan} \citep{astroplan2018},
	%\texttt{AstroImageJ} \citep{collins_astroimagej_2017},
  %\texttt{astropy} \citep{astropy_2018},
 % \texttt{astroquery} \citep{astroquery_2018},
  \texttt{astropy} \citep{astropy_2013,astropy_2018,astropy_2022},
  %\texttt{astroquery} \citep{astroquery_2018},
  %\texttt{BATMAN} \citep{kreidberg_batman_2015},
  %\texttt{ceres} \citep{brahm_2017_ceres},
  %\texttt{cdips-pipeline} \citep{bhatti_cdips-pipeline_2019},
  %\texttt{corner} \citep{corner_2016},
  %\texttt{emcee} \citep{foreman-mackey_emcee_2013},
  %\texttt{exoplanet} \citep{exoplanet:exoplanet}, and its
  %dependencies \citep{exoplanet:agol20, exoplanet:kipping13, exoplanet:luger18,
  % 	exoplanet:theano},
	%\texttt{gala} \citep{gala,PriceWhelan_2017_gala_zenodo},
	%\texttt{IDL Astronomy User's Library} \citep{landsman_1995},
  %\texttt{IPython} \citep{perez_2007},
	%\texttt{isochrones} \citep{morton_2015_isochrones},
	\texttt{lightkurve} \citep{2018ascl.soft12013L},
  %\texttt{matplotlib} \citep{hunter_matplotlib_2007}, 
  %\texttt{MESA} \citep{paxton_modules_2011,paxton_modules_2013,paxton_modules_2015}
  \texttt{numpy} \citep{2020Natur.585..357H}, 
  %\texttt{pandas} \citep{mckinney-proc-scipy-2010},
  \texttt{pyGAM} \citep{daniel_serven_2018_1208723},
  %\texttt{PyMC3} \citep{salvatier_2016_PyMC3},
  %\texttt{radvel} \citep{fulton_radvel_2018},
  %\texttt{scikit-learn} \citep{scikit-learn},
  \texttt{scipy} \citep{2020NatMe..17..261V},
  \texttt{TESS-point}  \citep{2020ascl.soft03001B},
  %\texttt{tesscut} \citep{brasseur_astrocut_2019},
  %\texttt{unpopular} \citep{hattorio_2021_cpm},
  %\texttt{VESPA} \citep{morton_efficient_2012,vespa_2015},
  %\texttt{webplotdigitzer} \citep{rohatgi_2019},
  \texttt{wotan} \citep{hippke_wotan_2019}.
}
\ 

\facilities{
 	{\it Astrometry}:
		Gaia \citep{2018A&A...616A...1G,2022arXiv220800211G}.
 	{\it Imaging}:
    Second Generation Digitized Sky Survey. %,
    %SOAR~(HRCam; \citealt{tokovinin_ten_2018}).
		%Keck:II~(NIRC2).
		%Gemini:South~(Zorro; \citealt{scott_nessi_2018}.
		%Gemini:North~(`Alopeke; \citealt{scott_nessi_2018,scott_twin_2021}.
 	{\it Spectroscopy}:
		%CTIO1.5$\,$m~(CHIRON; \citealt{tokovinin_chironfiber_2013}),
		%PFS ({\bf CITE}),
		%Tillinghast:1.5m~(TRES).
		%MPG2.2$\,$m~(FEROS; \citealt{kaufer_commissioning_1999}),
		%AAT~(Veloce; \citealt{gilbert_veloce_2018}).
		%AAT~(HERMES; \citealt{lewis_2002_hermers_2df,sheinis_2015_hermes}),
		Keck:I~(HIRES; \citealt{1994SPIE.2198..362V}).
		%VLT:Kueyen~(FLAMES; \citealt{pasquini_2002}).
 	  %Euler1.2m~(CORALIE),
 	  %ESO:3.6m~(HARPS; \citealt{mayor_setting_2003}).
 	{\it Photometry}:
		%ASTEP:0.40$\,$m (ASTEP400),
		%CTIO:1.0m (Y4KCam),
		%Danish 1.54m Telescope,
		%El Sauce:0.356$\,$m,
		%Elizabeth 1.0m at SAAO,
		%Euler1.2m (EulerCam),
		%Kepler,
		%Magellan:Baade (MagIC),
		%Max Planck:2.2m	(GROND; \citealt{greiner_grond7-channel_2008})
		%MuSCAT3 \citep{Narita_2020},
		%NTT,
		%SOAR (SOI),
 	  TESS \citep{2015JATIS...1a4003R},
		%TRAPPIST \citep{jehin_trappist_2011},
		%VLT:Antu (FORS2).
		%ZTF.
  {\it Broadband photometry}:
    2MASS \citep{2006AJ....131.1163S},
    APASS \citep{2016yCat.2336....0H},
		Gaia \citep{2018A&A...616A...1G,2022arXiv220800211G},
    SDSS \citep{2000AJ....120.1579Y},
    WISE \citep{2010AJ....140.1868W}.
}

\clearpage

\bibliographystyle{yahapj}                            
\bibliography{bibliography} 

%\clearpage

\startlongtable
\begin{deluxetable}{rrrrrrlrllrrrrrr}
\tabletypesize{\footnotesize}
%\tabletypesize{\scriptsize}
\tablecaption{Bona fide, candidate, and debunked complex quasiperiodic
  variables from the TESS 2-minute data. {\bf For internal
  review, full versions are available here
  \url{https://www.dropbox.com/scl/fo/twn4s9ckbevf75jqhtoy1/h?rlkey=t5cn8cx2uoc2ptdm9e570kp1b&dl=0}} \label{tab:thetable}}
%\toprule
%TIC & $T$ & $d$ & BP-RP & RUWE & $P$ & assoc & age & teff & rstar & mstar & rcr & qual & Nsectors \\
%\midrule
%\endhead
\startdata
  TIC & $T$ & $d$ & $\bprp$ & RUWE & $P$ & Assoc & Age & $T_{\rm eff}$ & $R_\star$ & $M_\star$ & $R$$_{\rm cr}$ & $P_{\rm sec}$ & Quality & Bin & $N_{\rm sector}$ \\
  -- &   mag & pc &    mag &      -- &        hr &   -- &        Myr &  K                  & $R_\odot$ & $M_\odot$ & $R_\star$         & hr                     & -- & -- & -- \\
\hline
368129164 & 9.29 & 18.3 & 2.9 & 6.95 & 6.44 & ABDMG & 149 & 3140 & 0.71 & 0.4 & 1.81 & 1 & 3 \\
405754448 & 9.63 & 96.5 & 1.75 & 4.66 & 12.92 & LCC & 15 & 4273 & 1.5 & 0.82 & 1.74 & 1 & 5 \\
167664935 & 10.31 & 63.0 & 2.52 & 1.62 & 14.05 & UCL & 16 & 3325 & 1.42 & 0.38 & 1.5 & 1 & 3 \\
311092148 & 11.03 & 26.9 & 3.04 & 1.31 & 7.86 & COL & 42 & 3035 & 0.5 & 0.27 & 2.6 & 1 & 1 \\
402980664 & 11.11 & 21.3 & 3.05 & 1.26 & 18.56 & COL & 42 & 3080 & 0.37 & 0.22 & 5.76 & 1 & 10 \\
50745567 & 11.28 & 38.5 & 3.23 & 1.84 & 6.34 & BPMG & 24 & 3014 & 0.67 & 0.28 & 1.68 & 1 & 2 \\
59836633 & 11.38 & 62.1 & 2.72 & 1.18 & 14.96 & BPMG & 24 & 3282 & 0.82 & 0.48 & 2.91 & 1 & 3 \\
425933644 & 11.4 & 44.3 & 2.83 & 2.29 & 11.67 & THA & 45 & 3151 & 0.63 & 0.4 & 3.06 & 1 & 6 \\
142173958 & 11.61 & 72.6 & 3.1 & 1.43 & 11.76 & TWA & 10 & 3028 & 1.15 & 0.26 & 1.45 & 1 & 3 \\
146539195 & 11.62 & 48.6 & 3.38 & 1.59 & 6.73 & BPMG & 24 & 2898 & 0.8 & 0.24 & 1.38 & 1 & 2 \\
206544316 & 11.63 & 43.1 & 2.89 & 1.27 & 7.73 & THA & 45 & 3114 & 0.57 & 0.35 & 2.44 & 1 & 6 \\
335598085 & 11.9 & 105.9 & 2.86 & 1.64 & 15.85 & LCC & 15 & 3119 & 1.34 & 0.28 & 1.56 & 1 & 3 \\
405910546 & 12.11 & 112.2 & 2.36 & 1.09 & 37.99 & LCC & 15 & 3455 & 0.92 & 0.6 & 5.26 & 1 & 4 \\
272248916 & 12.15 & 80.3 & 2.84 & 3.18 & 8.9 & UCL & 16 & 3193 & 0.81 & 0.4 & 1.97 & 1 & 3 \\
178155030 & 12.17 & 46.8 & 2.91 & 1.23 & 11.67 & THA & 45 & 3097 & 0.49 & 0.3 & 3.53 & 1 & 4 \\
224283342 & 12.29 & 38.1 & 3.04 & 1.24 & 21.3 & COL & 42 & 3050 & 0.39 & 0.22 & 6.08 & 1 & 3 \\
89026133 & 12.31 & 139.2 & 2.82 & 2.4 & 11.2 & UCL & 16 & 3188 & 1.33 & 0.31 & 1.29 & 1 & 3 \\
234295610 & 12.51 & 48.2 & 3.05 & 1.13 & 18.29 & THA & 45 & 3074 & 0.44 & 0.27 & 5.09 & 1 & 3 \\
118449916 & 12.54 & 69.2 & 3.1 & 8.17 & 12.31 & TAU & 2 & 3025 & 1.04 & 0.28 & 1.7 & 1 & 4 \\
67897871 & 12.55 & 145.5 & 3.02 & 1.67 & 6.23 & USCO & 10 & 3082 & 1.5 & 0.18 & 0.65 & 1 & 2 \\
353730181 & 12.65 & 107.3 & 2.75 & 1.17 & 13.51 & TAU & 2 & 3253 & 0.8 & 0.41 & 2.65 & 1 & 4 \\
201898222 & 12.68 & 42.2 & 3.22 & 1.14 & 10.7 & THA & 45 & 2996 & 0.39 & 0.2 & 3.69 & 1 & 5 \\
264767454 & 12.73 & 112.6 & 2.93 & 16.58 & 10.01 & COL(?) & 42 & 3150 & 1.0 & 0.42 & 1.76 & 1 & 13 \\
442571495 & 12.75 & 78.5 & 3.03 & 1.38 & 9.59 & UCL & 16 & 3099 & 0.65 & 0.3 & 2.35 & 1 & 3 \\
2234692 & 12.8 & 53.6 & 3.01 & 1.08 & 6.52 & COL & 42 & 3098 & 0.44 & 0.26 & 2.56 & 1 & 7 \\
94088626 & 12.88 & 57.8 & 3.06 & 1.1 & 6.6 & ARG & 45 & 3090 & 0.46 & 0.27 & 2.53 & 1 & 2 \\
264599508 & 12.88 & 80.0 & 3.0 & 1.53 & 7.9 & COL & 42 & 3098 & 0.62 & 0.4 & 2.4 & 1 & 7 \\
363963079 & 12.92 & 81.8 & 3.1 & 5.57 & 7.82 & ARG & 45 & 3040 & 0.67 & 0.4 & 2.21 & 1 & 7 \\
193831684 & 13.03 & 51.5 & 3.2 & 1.06 & 31.02 & BPMG & 24 & 2971 & 0.42 & 0.2 & 6.87 & 1 & 3 \\
177309964 & 13.1 & 91.1 & 2.93 & 1.31 & 10.88 & CAR & 45 & 3125 & 0.62 & 0.4 & 2.95 & 1 & 34 \\
425937691 & 13.18 & 43.8 & 3.79 & 1.69 & 4.82 & THA & 45 & 2782 & 0.41 & 0.16 & 1.91 & 1 & 5 \\
141146667 & 13.28 & 57.5 & 3.29 & 1.18 & 3.93 & FIELD & NaN & 2968 & 0.42 & NaN & NaN & 1 & 6 \\
332517282 & 13.29 & 39.1 & 3.29 & 0.93 & 9.67 & ABDMG & 149 & 2975 & 0.28 & 0.2 & 4.87 & 1 & 3 \\
144486786 & 13.3 & 70.4 & 3.07 & 7.62 & 6.82 & COL & 42 & 3074 & 0.51 & 0.3 & 2.38 & 1 & 4 \\
38820496 & 13.3 & 44.1 & 3.38 & 1.21 & 15.73 & THA & 45 & 2903 & 0.34 & 0.16 & 5.13 & 1 & 5 \\
289840926 & 13.31 & 40.2 & 3.76 & 1.0 & 4.8 & BPMG & 24 & 2807 & 0.36 & 0.14 & 2.05 & 1 & 3 \\
404144841 & 13.33 & 76.0 & 3.2 & 1.03 & 10.74 & TWA & 10 & 3008 & 0.52 & 0.22 & 2.86 & 1 & 4 \\
89463560 & 13.45 & 126.9 & 2.97 & 1.31 & 9.43 & ARG & 45 & 3055 & 0.75 & 0.37 & 2.15 & 1 & 10 \\
300651846 & 13.49 & 109.6 & 2.87 & 1.25 & 8.26 & CAR & 45 & 3136 & 0.62 & 0.4 & 2.44 & 1 & 31 \\
267953787 & 13.49 & 130.6 & 3.58 & 1.09 & 17.46 & TAU & 2 & 2826 & 1.06 & 0.12 & 1.59 & 1 & 4 \\
68812630 & 13.6 & 132.7 & 3.23 & 1.24 & 9.04 & TAU & 2 & 2996 & 0.76 & 0.27 & 1.86 & 1 & 3 \\
141306513 & 13.65 & 50.0 & 3.42 & 1.23 & 13.36 & THA & 45 & 2964 & 0.32 & 0.16 & 4.85 & 1 & 2 \\
201789285 & 14.03 & 45.2 & 3.85 & 1.18 & 3.64 & THA & 45 & 2757 & 0.3 & 0.12 & 2.02 & 1 & 5 \\
294328887 & 14.23 & 97.8 & 3.23 & 1.11 & 8.51 & CARN & 200 & 2994 & 0.45 & 0.35 & 3.33 & 1 & 35 \\
312410638 & 14.3 & 138.8 & 3.13 & 1.11 & 28.06 & UCL & 16 & 3030 & 0.58 & 0.25 & 5.11 & 1 & 3 \\
38539720 & 14.52 & 132.1 & 3.38 & 1.27 & 9.16 & PERI & 120 & 2924 & 0.57 & 0.25 & 2.43 & 1 & 1 \\
359892714 & 14.53 & 94.0 & 3.99 & 1.18 & 11.33 & EPSC & 3 & 2675 & 0.55 & 0.13 & 2.36 & 1 & 6 \\
118769116 & 14.58 & 120.3 & 3.61 & 1.11 & 8.56 & TAU & 2 & 2852 & 0.56 & 0.2 & 2.18 & 1 & 4 \\
440725886 & 14.69 & 131.6 & 2.94 & 1.23 & 3.92 & PLE & 112 & 3109 & 0.45 & 0.35 & 1.99 & 1 & 5 \\
397791443 & 15.01 & 149.5 & 3.07 & 1.05 & 6.95 & IC2602 & 46 & 3031 & 0.48 & 0.26 & 2.46 & 1 & 6 \\
160329609 & 9.65 & 8.7 & 3.41 & 1.28 & 24.31 & ARG & 45 & 2912 & 0.35 & 0.16 & 6.6 & 0 & 3 \\
148646689 & 12.14 & 142.4 & 2.44 & 2.24 & 10.63 & UCL & 16 & 3466 & 1.25 & 0.55 & 1.61 & 0 & 3 \\
280945693 & 12.27 & 98.3 & 2.97 & 1.1 & 15.27 & LCC & 15 & 3103 & 1.09 & 0.31 & 1.94 & 0 & 5 \\
165184400 & 12.37 & 43.1 & 3.03 & 1.14 & 15.91 & THA & 45 & 3076 & 0.42 & 0.25 & 4.74 & 0 & 4 \\
245834739 & 12.55 & 110.5 & 2.85 & 1.25 & 10.47 & TAU & 2 & 3112 & 1.02 & 0.36 & 1.68 & 0 & 6 \\
125843782 & 13.01 & 128.5 & 2.87 & 0.99 & 44.17 & TAU & 2 & 3135 & 0.9 & 0.35 & 4.97 & 0 & 4 \\
244161191 & 13.17 & 44.4 & 3.54 & 1.33 & 7.17 & COL & 42 & 2860 & 0.38 & 0.18 & 2.8 & 0 & 3 \\
231058925 & 13.17 & 51.1 & 3.26 & 1.15 & 8.87 & THA & 45 & 2978 & 0.38 & 0.2 & 3.33 & 0 & 5 \\
301676454 & 13.4 & 71.1 & 3.07 & 1.35 & 9.18 & ARG & 45 & 3009 & 0.47 & 0.25 & 2.96 & 0 & 1 \\
58084670 & 13.58 & 140.6 & 2.81 & 1.04 & 11.16 & FIELD & NaN & 3138 & 0.77 & NaN & NaN & 0 & 6 \\
67745212 & 13.63 & 27.7 & 3.82 & 1.09 & 5.12 & COL & 42 & 2781 & 0.21 & 0.09 & 3.17 & 0 & 2 \\
5714469 & 13.73 & 78.4 & 3.66 & 1.07 & 10.35 & UCL & 16 & 2828 & 0.54 & 0.18 & 2.53 & 0 & 3 \\
259586708 & 13.82 & 96.3 & 2.93 & 1.22 & 22.52 & COL & 42 & 3133 & 0.46 & 0.29 & 5.84 & 0 & 7 \\
435903839 & 11.95 & 92.4 & 2.48 & 11.63 & 10.82 & ABDMG(?) & 149 & 3458 & 0.76 & 0.54 & 2.66 & -1 & 6 \\
57830249 & 11.96 & 48.7 & 3.21 & 1.27 & 43.82 & TWA & 10 & 2948 & 0.7 & 0.25 & 5.63 & -1 & 3 \\
193136669 & 13.06 & 61.3 & 3.51 & 1.08 & 37.64 & TWA & 10 & 2855 & 0.58 & 0.19 & 5.62 & -1 & 4 \\

\enddata
\tablecomments{This table includes \ngoods\ CQVs (\texttt{Quality}
  flag 1), \nmaybes\ ambiguous CQVs (\texttt{Quality} flag 0), and
  \ndebunked\ impostors (\texttt{Quality} flag -1).  The three-bit
  binarity flag ``Bin'' is for Gaia DR3
  \texttt{radial\_velocity\_error} outliers (bit 1), Gaia DR3
  \texttt{ruwe} outliers (bit 2), and stars with multiple TESS periods
  (bit 3).  The machine-readable version, available online, includes
  additional columns for the Gaia DR2 and DR3 source identifiers, as
  well as the stellar parameter uncertainties.  The age uncertainties
  are typically $\approx \pm 10\%$, but can be asymmetric.  {\bf The
  median temperature, radius, and mass uncertainties are $\pm XXX$\,K,
  $\pm XX$\%, and $\pm YY$\% respectively}.  $N_{\rm sector}$ denotes
  the number of TESS sectors for which {\it any} data are expected to
  be acquired between July~2018 and Oct~2024.  This number is
  generally greater than the number of sectors for which 120-second
  cadence data exist.   Association names and provenance follow
  conventions adopted by \citet{2018ApJ...856...23G}: 
	ABDMG: AB~Doradus moving group \citep{2015MNRAS.454..593B}.
	ARG: Argus \citep{2019ApJ...870...27Z}.
	%APER: $\alpha$~Persei open cluster \citep{2023AJ....166...14B}.
	BPMG: $\beta$~Pic moving group \citep{2015MNRAS.454..593B}.
  CARN: Carina Near moving group \citep{2006ApJ...649L.115Z}.
	COL: Columba \citep{2015MNRAS.454..593B}.
	EPSC: $\epsilon$ Chamaeleontis \citep{2013MNRAS.435.1325M}.
	LCC: Lower Centaurus Crux \citep{2016MNRAS.461..794P}.
	PERI: Pisces-Eridani \citep{2019AJ....158...77C}.
	PLE: Pleiades \citep{2015ApJ...813..108D}.
	TAU: Taurus \citep{1995ApJS..101..117K}.
	THA: Tucana-Horologium assocation \citep{2015MNRAS.454..593B}.
	TWA: TW Hydrae assocation \citep{2015MNRAS.454..593B}.
	UCL: Upper Centaurus Lupus \citep{2016MNRAS.461..794P}.
	USCO: Upper Scorpius \citep{2016MNRAS.461..794P}.
	The ``(?)'' string denotes low-confidence membership.
}
\end{deluxetable}
\clearpage

\appendix

%\section{Validation plots}
%\label{app:vetting}
%
%Figure~\ref{fig:vet} shows the type of plot used to visually assess
%whether a source was likely to be a CQV, eclipsing binary, or simply a
%rapidly rotating star.
%
%\begin{figure*}[!t]
%	\begin{center}
%    \centering
%    \includegraphics[width=0.96\textwidth]{annotated_402980664_S0018_120sec_cpvvetter.pdf}
%		\vspace{-0.45cm}
%		\caption{
%      {\bf Validation plots used to manually label CQVs}.  The complete figure
%      set, with one image per sector for each of \ncpvsfound\ CQVs
%      and candidate CQVs
%      is available online {\bf For internal collaboration review:
%      \url{https://www.dropbox.com/scl/fo/zlj3txot4cvymfb22wewu/h?dl=0&rlkey=3ec5f9o5xewrixzfkhkdenopa}}.
%      Panels are as follows.
%      {\it a)}: Phase-folded light curve; gray points are raw 2-minute
%      data and black points are binned to 200 points per cycle.
%      {\it b)}: Phase-dispersion minimization (PDM) periodogram.
%      Dotted lines show up to the 10$^{\rm th}$ harmonic and
%      subharmonic.
%      {\it c)}: DSS finder chart, with 1- and 2-TESS pixel radius
%      circles displayed for scale.
%      {\it d)}: Cleaned light curve, binned to 20-minute cadence, in
%      Barycentric TESS Julian Date (BTJD).
%      {\it e)}: Phase-folded light curve, binned to 100 points per
%      cycle.  The gray line denotes the automated spline-fit to the
%      wrapped phase-folded light curve, and small gray triangles
%      denote automatically identified local minima.
%      {\it f)}: Phase-folded light curve at twice the peak period.
%      {\it g)}: Phase-folded light curve at half the peak period.
%      {\it h)}: Phase-folded time-series within the ``background''
%      aperture defined in the SPOC light curves.
%      {\it i)}: Phase-folded flux-weighted centroid in the column
%      direction.
%      {\it j)}: Phase-folded flux-weighted centroid in the row
%      direction.
%      {\it k)}: Gaia DR2 color--absolute magnitude diagram.     
%      The gray background denotes stars within 100\,pc.
%      {\it l)}: Information from Gaia DR2, TIC8, and the automated
%      dip-counting search pipeline.  ``Neighbors'', abbreviated
%      ``nbhr'', are listed within apparent distances of 2 TESS pixels
%      if $\Delta T$$<$2.5.
%      {\it m)}: BANYAN-$\Sigma$ v1.2 association probabilities, calculated
%      using positions, proper motions, and the parallax.
%      }
%		\label{fig:vet}
%	\end{center}
%\end{figure*}



\section{TIC~300651846}
\label{app:tic3006}

Figures~\ref{fig:tic3006timegroups} and~\ref{fig:tic3006river} show
120-second cadence data for TIC~300651846, a CQV in the TESS
continuous viewing zone.  If it were not for the existence of
TIC~402980664, this source would probably have received greater
attention.  With the exception of a few sectors, TESS data will exist
for TIC~300651846 for at least Sectors 1-12, 27-39, and 61-69.  While
most of the available data will exist in the full frame images, in
Figures~\ref{fig:tic3006timegroups} and~\ref{fig:tic3006river} focus
only on the currently available 120-second cadence data.

During Sectors 32-39, the source shows between one and four local
minima per cycle.  During the early portions of Sectors 61-65, it is
somewhat more complex, with at least five clear local minima per
cycle.  As the source evolves, this complexity decreases, while the
sharpness of one of the minima appears to increase.  We did not find
evidence for any obvious ``state-switches'' analogous to those that we
observed in LP 12-502; gradual evolution over timescales of
$\approx$50-100 cycles seem to be the norm for TIC~300651846.  Unlike
the TIC~402980664 river plots (Figure~\ref{fig:lpriver0}), we did not
subtract any ``continuum sinusoid'' for this source, because the
continuum is not as obviously defined.

\begin{figure*}[!t]
	\begin{center}
		\subfloat{
			\includegraphics[width=0.48\textwidth]{f11a.pdf}
			\includegraphics[width=0.48\textwidth]{f11b.pdf}
		}
	\end{center}
	\vspace{-0.4cm}
	\caption{
		{\bf Light curve evolution of TIC 300651846}.
    All available 120-second cadence data as of 2023 Aug 11 are shown.
    Cycles 0 to 622 span TESS Sectors 32-39 (Nov 2020--June 2021);
    cycles 2296-2676 span Sectors 61-65 (Jan--June 2023).  We assumed
    a 8.254\,hr period and a fixed reference epoch (BTJD 2174.127) for
    both panels.  Light curve segments are split based on the presence
    of gaps longer than three hours.  Cycle numbers are listed in the
    lower-right of each light curve segment.
	}
	\label{fig:tic3006timegroups}
\end{figure*}


\begin{figure*}[!t]
	\begin{center}
		\subfloat{
			\includegraphics[width=0.48\textwidth]{f12a.pdf}
			\includegraphics[width=0.48\textwidth]{f12b.pdf}
		}
	\end{center}
	\vspace{-0.4cm}
	\caption{
		{\bf River plots of TIC 300651846}.
    This is an alternative visualization of the data in
    Figure~\ref{fig:tic3006timegroups}.  All available 120-second
    cadence data as of 2023 Aug 11 are shown.  Cycles 0 to 622 span
    TESS Sectors 32-39 (Nov 2020--June 2021); cycles 2296-2676 span
    Sectors 61-65 (Jan--June 2023).  We assumed $P$$=$8.254\,hr and
    $t_0$=2174.127 [BTJD].  Note that the two panels have slightly
    different color scales.
	}
	\label{fig:tic3006river}
\end{figure*}



\section{No significant power at 20~second cadence}

TESS was the first instrument to show that CQV light curves contain
power at timescales of a few minutes
\citep{2019A&A...625L..13Z,2022AJ....163..144G}.  This advance was
enabled by the fifteen-fold faster cadence in the TESS 2-minute data,
relative to K2.  A logical follow-up is to ask whether the periodic
components of the CQV light curves contain power at timescales below
one minute.  Between 2020 and 2021, we observed 10 CQVs at 20-second
cadence with TESS in order to explore this question (TESS DDT029).
The stars were TICs 142173958, 146539195, 24518895, 276453848,
264599508, 363963079, 144486786, 408188366, 300651846, 262400835.
These sources were selected from CQVs known at the time to have short
periods and sharp features when observed at 2-minute cadence.
Comparing the 20-second to 120-second data for these stars (data
available on MAST), we concluded that these CQVs did not contain
appreciable power at timescales shorter than a few minutes.

% \section{Chromaticity in TIC 262400835}
% 
% TIC~262400835 ($d$=174\,pc) is formally outside the scope of the
% current work.  However, this CQV was observed using MuSCAT2 on 2020
% December 12, 13, and 16, and the results might be worth including in
% this study.
% {\bf todo: describe observations / decide whether to include!}.
% 
% We include {\bf a table of the photometry} here to enable potential
% future deeper analyses of the chromaticity of this object class.
% 
% Generally, these data serve as a minor addition beyond the
% observations that have been acquired by
% \citet[e.g.][]{2017PASJ...69L...2O,2020PASJ...72...23T,2022AJ....163..144G,2023MNRAS.518.2921K}
% on this topic.  \citet{2023MNRAS.518.2921K} provides what we find to
% be the most lucid summary, and we quote: ``amplitudes are almost
% always larger, the shorter the wavelength of the filter, but the
% relationship can be weak or non-monotonic.''
% 
% \begin{figure*}[!t]
% 	\begin{center}
%     \includegraphics[width=0.45\textwidth]{TIC262400835_multicolor_phase_stacked.pdf}
%     	\end{center}
%     \vspace{-0.4cm}
% 		\caption{
% 	      {\bf Chromaticity in TIC~262400835}.
% 		}
% 		\label{fig:muscat}
% \end{figure*}



\clearpage
\listofchanges


\end{document}
